\documentclass[12pt]{article} %,twoside,a4paper

%\usepackage[a4paper,margin=24.5mm]{geometry} %margin=24.5mm
\usepackage[left=2.5cm,right=2.5cm,top=2.5cm, bottom=3.3cm]{geometry}
%\usepackage{geometry}

\usepackage[utf8]{inputenc}
\usepackage[english,catalan]{babel}
%%%\usepackage[english,spanish]{babel}
\usepackage{natbib}
\usepackage{float} 
\usepackage{graphicx}


% Establim el format de la pàgina:
%\usepackage{fancyhdr}

%\setlength{\parskip}{0.2cm}
%\usepackage[T1]{fontenc}
%\usepackage{newunicodechar}
%\newunicodechar{Ŀ}{\L.}
%\newunicodechar{ŀ}{\l.}
\usepackage{siunitx}
\usepackage{amsmath}
%\usepackage{adjustbox}
%\usepackage{tabularx}
\usepackage{mathtools}
\usepackage{color}
\usepackage{caption}
\usepackage{subcaption}
\usepackage[colorlinks=true,urlcolor=blue,linkcolor=blue,citecolor=blue]{hyperref}
%\usepackage{booktabs} %Publication quality tables in LaTeX.
%\usepackage{pdflscape}
\usepackage{attachfile}
\usepackage{todonotes}
%\usepackage{pdfpages}

\usepackage{sagetex}
%\definecolor{gris}{RGB}{220,220,220}
\let\vec\mathbf %vectors en negreta en compte de arrow

\author{Casimiro Victoria Castillo}


\begin{document}

\newgeometry{margin=1in}
\begin{titlepage}
   \noindent\raisebox{0pt}[0pt][0pt]{
	\includegraphics[width=10.5cm]{./portada-TFG-LaTeX/marca-Facultat-Fisica-UV-1-linia.pdf}}\par
   \vspace{8.5cm}
   % Titles
   {\centering
      \scalebox{1.125}{\bfseries\sffamily\Huge
		Treball de Fi de Grau en Física}\par
      \rule{16.13cm}{1.5pt}\par
      \vspace{4.5cm}
      {\bfseries\sffamily\LARGE Fonons i espectroscòpia Raman en semiconductors bidimensionals}\par
   }
   \vfill
   {\raggedleft\sffamily
		AGOST de 2021\par
      \vspace{\baselineskip}
		Alumne/a: Casimiro Victoria Castillo\par
      \vspace{\baselineskip}
		Tutor/a (1): Alberto García Cristóbal\par
		Tutor/a (2): Alejandro Molina Sanchez\par
   }
\end{titlepage}
\restoregeometry

%%%%%%%%%%%%%%%%%%%%%%%%%%%%%%
%%% Abstract
%%%%%%%%%%%%%%%%%%%%%%%%%%%%%%

%\begin{abstract}
\subsection*{Resum / \emph{Abstract}}
 
 Els materials bidimensionals ($2D$) com el grafè són de gran interès tant per les seues propietats físiques exclusives com per les seues aplicacions potencials. L'estudi de la dinàmica de la xarxa cristal·lina (\textit{fonons}) d'estos materials és un requisit previ per entendre la seua estabilitat estructural i propietats tèrmiques, així com les seues propietats de transport i òptiques. 
 
 En este treball es calculen els modes vibracionals  d'un  material semiconductor 2D, el nitrur de bor monocapa. Emprant un model clàssic i senzill, el model de constants de força, s'obtenen les expressions analítiques per als modes de vibració als punts crítics de la zona de Brillouin. Evidentment, un model tan senzill com l'emprat, on sols es consideren les interaccions amb uns pocs veïns, i on s'assumeix a més una forma diagonal per al tensor de constants de força, té moltes limitacions, però ajustant les constants de força emprant la relació de dispersió calculada a partir de primers principis mitjançant el mètode de \emph{Density Functional Perturbation Theory} observem que, almenys, obtenim resultats acceptables per a baixes freqüències.

\vspace{\baselineskip}

\begin{otherlanguage}{english}\itshape

Two-dimensional materials ($2D$) like graphene are of great interest for both their unique physical properties and their potential applications. The study of the dynamics of the crystal lattice (\textit{phonons}) of these materials is a prerequisite for understanding their structural stability and thermal properties, as well as their transport and optical properties.
 
 In this work the vibrational modes of a 2D model semiconductor material, the monolayer boron nitride, are calculated. Using a classical and simple model, the model of force constants, analytical expressions for the modes at the critical points of the Brillouin zone are obtained. Obviously, a model as simple, where only interactions with a few neighbors are considered and where a diagonal shape is assumed for the force constant tensor, has many limitations, however by fitting the force-constants to the ab initio  dispersion relation, calculated using the method of \emph{Density Functional Perturbation Theory} we observe that, at least, we obtain acceptable results for low frequencies.

\noindent 

\end{otherlanguage}

%\textit{Nota: s'ajunta el codi font del document: \attachfile[author=Casimir]{TFG-Casimir.tex}}

%\newpage
%\end{abstract}

\section{Introducció teòrica}

En este treball s'empra el mètode de constants de força (\emph{FCM}, pel seu acrònim en anglès, \emph{\foreignlanguage{english}{force-constant method})}, per descriure la dinàmica de la xarxa cristal·lina d'un material bidimensional, el nitrur de bor monocapa, \emph{BN}. La metodologia emprada per realitzar els càlculs es basa sobretot en els articles recomanats pels tutors, en particular \cite{wirtz04_phonon_disper_graph_revis} i \cite{falkovsky08_symmet_const_phonon_disper_graph}, tot i que en estos articles es tracta el grafè, i en el present treball estudiem el nitrur de bor.

En este model la dinàmica dels àtoms es descriu considerant que cada àtom interacciona amb els del seu entorn fins a un determinat nombre de veïns (els veïns es classifiquen segons la seua distància), caracteritzant les interaccions amb els veïns mitjançant un tensor de constants de forces, una manera visual de representar el model seria considerar que els àtoms estan connectats per molls. Altres mètodes, com els càlculs realitzats a partir de primers principis (\emph{ab initio}) emprant la teoria del funcional densitat (DFT, pel seu acrònim en anglès \foreignlanguage{english}{density-functional density}) o el mètode de camp de força de valència (\emph{VFF}, \foreignlanguage{english}{valence force field}) necessiten temps de càlcul molt més llargs (i donen resultats molt més precisos, clar). El mètode de constants de força empra un conjunt reduït de paràmetres que poden ajustar-se a les mesures experimentals o bé, com s'ha fet en este treball, a la relació de dispersió calculada per primers principis, pel mètode de \emph{Density Functional Perturbation Theory} en este treball.  Notem que encara que estem tractant amb un mètode senzill, este ha demostrat que pot proporcionar resultats fiables \cite{wirtz04_phonon_disper_graph_revis}.

Abans d'explicar amb detall el mètode emprat en este treball repassem alguns conceptes i aclarim la notació emprada.

\subsection{Xarxa cristal·lina}

Sabem que els àtoms que constitueixen un sòlid cristal·lí estan distribuïts d'una manera regular en l'espai, i per descriure esta distribució regular s'empra el concepte matemàtic de \emph{xarxa cristal·lina}, introduït per A. Bravais en 1845: una xarxa cristal·lina es defineix com una distribució discreta i regular de punts (\emph{nucs}) que té sempre la mateixa aparença independentment de quin punt escollim com origen de coordenades.

En el cas que anem a tractar, un cristall bidimensional, la xarxa cristal·lina bidimensional pot generar-se a partir de dos vectors base, de manera que els vectors de posició dels \emph{nucs de la xarxa}, els vectors de translació cristal·lina,  tenen la forma:
\begin{equation}
\label{eq:R_l}
 \vec R_\vec l=\vec R_{l_1 l_2}=l_1 \vec a_1 +l_2\vec a_2
\end{equation}

on $\vec l$ es denomina índex del nuc (notem que en la literatura sobre el tema sol emprar-se com a índex la lletra $n$, però crec que un índex vectorial especifica millor els nucs). Si tots els nucs de la xarxa tenen índexs enters llavors els vectors base s'anomenen \emph{primitius}.

Els cristalls o \emph{estructures cristal·lines} són distribucions regulars d'àtoms en l'espai de posicions i els podem descriure associant a cadascun dels nucs d'una xarxa cristal·lina en l'espai de posicions un conjunt d'àtoms o base (matemàticament és el producte de \emph{convolució} de la base i la xarxa de nucs). En el cas del \emph{BN} tractem amb una base diatòmica, formada per un nucli de bor, $B$, i un de nitrogen, $N$. 

En dues dimensions, els paral·lelograms amb els quals podem omplir completament l'espai per translació cristal·lina i que contenen almenys un nuc de la xarxa es coneixen com cel·les unitat. La cel·la unitat més senzilla és la que té per costats el vectors base, i es coneixen com \emph{cel·les de Bravais}.

\subsubsection{Xarxa recíproca. Primera zona de Brillouin.}
Donat un conjunt de vectors base, $p_i$, de la xarxa cristal·lina en l'espai de posicions, la condició:

\begin{equation}
\label{eq:rec1}
\vec p_i\cdot\vec p_j^{*}=\delta_{ij}
\end{equation}

on $\delta_{ij}$ és la delta de Kronecker, defineix un altre conjunt de vectors $p_j^*$, coneguts com vectors recíprocs, i què són els vectors base que defineixen una altra xarxa coneguda com \textit{xarxa recíproca}. Els  vectors base recíprocs, i els vectors de translació cristal·lina recíprocs, tenen dimensions de inversa de longitud i es representen en l'\textit{espai recíproc} o de nombres d'ona. \textit{Les xarxes cristal·lines real i recíproca són dues descripcions equivalents del mateix sistema físic: el sòlid cristal·lí} que s'està estudiant.

Podem interpretar que
\begin{equation}
\label{eq:rec3}
\vec K_{h_1h_2}=2\pi\left(h_1\vec a_1^{*}+h_2\vec a_2^{*}\right)
\end{equation}

són els vectors de translació cristal·lina que defineixen una xarxa cristal·lina en l'espai $2\pi$-recíproc (sols es diferencia de l'espai recíproc en un factor d'escala $2\pi$). En termes físics, l'espai $2\pi$-recíproc és l'\emph{espai de vectors d'ona} $\vec k$, i a falta d'un factors d'escala $\hbar$, coincideix amb l'espai de moments  $\vec p=\hbar\vec k$.

Cal tenir present que l'espai $2\pi$-recíproc és el fonamental en l'estudi dels sòlids cristal·lins, ja que els estats de les partícules i les interaccions físiques de interès es descriuen en l'espai de vectors d'ona, $\vec k$.


\paragraph{Cel·les de Wigner-Seitz (WS) i primera zona de Brillouin (ZB)}

Per descriure la xarxa $2\pi$-recíproca, s'empra el criteri de \textit{Wigner-Seitz}. Les \textit{cel·les de Wigner-Seitz} estan centrades en un nuc de la xarxa i es defineixen com la regió més pròxima a un nuc (el del centre de la cel·la) que a qualsevol altre. Per determinar la seua forma, partim d'un nuc qualsevol com a origen, construïm els segments que uneixen este nuc amb els seus veïns i es tracen els plans que bisecten cadascun d'estos segments: la cel·la de \emph{WS} és la cel·la de menor volum al voltant de l'origen que està delimitada per estos plans (rectes en el cas d'una xarxa bidimensional).

Notem que en l'espai de $\vec k$ s'empren ce\l.les unitat de \emph{WS} mentre que en l'espai de posicions sempre emprem ce\l.les unitat de Bravais.
La ce\l.la de WS de la xarxa $2\pi$-recíproca es coneix com \textbf{primera zona de Brillouin} (ZB)


\subsection{Vibracions atòmiques en cristalls}
Passem ara a descriure el model emprat per descriure les vibracions dels àtoms del cristall.
\subsubsection{Model de BORN i VON KARMAN}
Els àtoms constituents de un sòlid cristal·lí no estan immòbils sinó que vibren al voltant de la seua posició de equilibri, amb una amplitud que depèn de la temperatura 

En 1912, Born y Von Karman \cite{Born:1912:SRG} introduïren un model per explicar la dinàmica cristal·lina, on la idea fonamental és que cada mode normal té l'energia de un oscil·lador de Planck.

Assumiren que els àtoms es troben disposats en una matriu tridimensional periòdica, i que la força sobre un àtom depèn no del desplaçament d'este respecte de la seua posició d'equilibri, sinó del desplaçament relatiu d'este àtom respecte als seus veïns. Introduïren també les condicions de contorn periòdiques que simplifiquen considerablement els càlculs. 
 
La dinàmica del sistema es descriu de manera senzilla, no en termes de les vibracions de àtoms individuals, sinó en termes de moviments col·lectius en forma de ones viatgeres anomenades vibracions cristal·lines \emph{(\foreignlanguage{english}{lattice vibrations})} per Born. Cada vibració cristal·lina es caracteritza per una freqüència i un vector de ona.
La vibració cristal·lina quantitzada, o mode normal, s'anomena fonó per analogia amb el fotó. %i te propietats de quasi-partícula

Un fonó és un mode normal de vibració quantitzat que té lloc en una xarxa cristal·lina.  Aquests modes normals són importants perquè qualsevol moviment vibracional de la xarxa pot descriure's com una superposició de modes normals de distinta freqüència, en este sentit són la base de les vibracions de la xarxa. Els fonons tenen una gran importància en moltes propietats físiques dels sòlids. 

%Els fonons son una version mecano-quàntica dels modes normals de vibració de la mecànica clàssica, on cada par de la xarxa osci\l.la amb la mateixa freqüència.

\subsubsection{Aproximació adiabàtica}

Moltes propietats físiques dels sòlids poden classificar-se bé com electròniques o bé com vibracionals, segons estiguen determinades pels electrons (de valència) o per les vibracions dels àtoms (nuclis iònics): normalment considerem els nuclis i els electrons com a constituent independents del sòlid, ja que les masses dels electrons i dels nuclis són tan diferents que el moviment dels nuclis és molt més lent que el dels electrons. L'\emph{aproximació adiabàtica}, introduïda per Born i Oppenheimer \cite{ANDP:ANDP19273892002} estableix que els electrons responen de manera pràcticament instantània als desplaçaments atòmics, de manera que el seu estat ve determinat per les posicions atòmiques instantànies, mentre que els àtoms no poden respondre a les ràpides variacions espacials dels electrons: diguem que els electrons segueixen el moviment iònic adiabàticament.
%els electrons no realitzen transicions abruptes de un estat a altres estats, si no que un estat electrònic es deforma progressivament degut als desplaçaments iònics.

Notem, però, que estem tractant amb una primera aproximació al problema i algunes propietats físiques venen determinades per la interacció entre els electrons i les vibracions atòmiques.

A continuació es mostra una breu descripció d'esta aproximació, basada en l'explicació que trobem a \cite{brueesch82_phonon}
La idea essencial es que donat els ions són entre $2\times 10^3$ a $10^5$ vegades més pesats es mouen molt més lentament que els electrons.

Considerem el Hamiltonià total per al cristall, denotat per $r$ com el símbol col·lectiu per a les coordenades dels electrons de valència i $R$ per a les coordenades dels ions. Podem escriure:
\begin{equation}
\label{eq:hamiltonià}
 H=T_e+T_i+V_{ii}(R)+V_{ee}(r)+V_{ie}(r,R)
\end{equation}

on els subscripts $i$ i $e$ denoten ions i electrons de valència, respectivament. $T_e$ i $T_i$ són els operadors d'energia cinètica dels electrons i ions, i $V_{ee}(r)$ i $V_{ii}(R)$ són les energies potencials de interacció electró-electró i ió-ió, respectivament, mentre que $V_{ie}(r,R)$ dóna compte de l'energia potencial deguda a la interacció electró-ió. Denotant per $\Xi(r,R)$ la funció propia del Hamiltonià (\ref{eq:hamiltonià}),

\begin{equation}
\label{eq:HXi}
 H\Xi(r,R)=E\Xi(r,R)
\end{equation}

on $E$ és l'energia total del sistema sencer. En l'aproximació adiabàtica els ions es troben fixes en una certa configuració, i per aquest particular $R$ podem escriure l'equació de Schrödinger per als electrons com:

\begin{equation}
\label{eq:Eechi}
 \left[T_e+V_{ee}(r) + V_{ie}(r,R)\right]\chi_R(r)=E_e(R)\chi_R(r)
\end{equation}

on $\chi_R(r)$ es la funció d'ona per al sistema complet d'electrons; $\chi_R(r)$ es una funció de totes les variables representades per $r$ i conté les coordenades iòniques $R$ com a paràmetres. Per tant, l'energia electrònica $E_e(R)$ depèn de les coordenades iòniques, $R$.
Si considerem que la funció d'ona és separable:

\begin{equation}
\label{eq:Xichipsi}
 \Xi(r,R)=\chi_R(r)\psi(R)
\end{equation}

Substituïnt \ref{eq:Xichipsi} en \ref{eq:HXi} i emprant \ref{eq:Eechi}, pot comprovar-se que \ref{eq:Xichipsi} es també solució de \ref{eq:HXi} sempre que certs termes puguen negligir-se i la funció dóna $\psi(R)$ complisca:

\begin{equation}
 \left[T_i+V_{ii}(R)+E_e(R)\right]\psi(R)=E \psi(R)
\end{equation}

L'equació involucra únicament la funció d'ona dels ions. Per tant, que per al moviment iònic podem considerar una \emph{energia potencial efectiva}

\begin{equation}
 V(R)=V_{ii}(R)+E_e(R)
\end{equation}

que té en compte la contribució dels electrons a traves de $E_e(R)$. L'aproximació adiabàtica fa possible treballar amb un potencial efectiu $V(R)$ quan tractem amb el moviments dels ions, el qual depén de totes les coordenades $\vec R_{\vec l, \alpha}$ dels ions: $V(R)=V(\ldots R_l\ldots R_k\ldots)$

A continuació es mostra  com se simplifica encara més el problema si considerem que el desplaçament de cada ió respecte de la seua possició d'equilibri, $\vec u_{\vec l,\alpha}=\vec R_{\vec l,\alpha}-\vec R_{\vec l,\alpha}^{(0)}$ és menuda.
  
\subsubsection{Aproximació harmònica}

Les vibracions reticulars estan regides per les forces que experimenten els àtoms quan es desplacen de la seua posició d'equilibri. 

La primera hipòtesi és que cada àtom té una posició d'equilibri en el cristall, que denotem per $\vec R^{(0)}_{\vec l,\vec\alpha}=\vec R_\vec l+\vec d_\alpha$, on $\vec R_\vec l$ recordem que és el vector de posició del nuc, o cel·la, $\vec l$, donat per l'equació (\ref{eq:R_l}), i $\vec d_\alpha$ ens dóna la posició de cadascun dels àtoms de la base respecte a l'origen de la cel·la. 

Considerarem que estos àtoms vibren al voltant d'esta posició d'equilibri, $\vec R_{\vec l,\vec\alpha}=\vec R^{(0)}_{\vec l,\alpha}+\vec u_{\vec l,\alpha}(t)$, amb una amplitud menuda (en comparació amb la distància interatòmica) al voltant d'aquesta,   de manera que el sòlid es troba en estats que corresponen al que macroscòpicament es coneix com \textit{la regió de comportament elàstic lineal}, on es verifica la llei de Hooke.

Podem per tant aproximar l'energia potencial de interacció pel terme harmònic del seu desenvolupament en sèrie de potencies del desplaçament, $\vec u_{\vec l,\alpha}=\vec R_{\vec l,\alpha}-\vec R^{(0)}_{\vec l,\alpha}$:

\begin{equation}
 V=\frac{1}{2}\sum_{\vec l',\alpha'}\sum_{\vec l,\alpha}\vec u_{\vec l'\alpha'}\cdot\vec u_{\vec l,\alpha}
\end{equation}

Notem que el terme d'ordre $0$ del desenvolupament en sèrie de Taylor correspon a un nombre arbitrari (què escollim com $0$ per conveniència) i el terme de primer ordre tambè l'escollim nul, ja que el desenvolupament el realitzem escollint el sistema en l'equilibri a temperatura nul·la, i en este estat l'energia total es mínima. El termes de major ordre els negligim.

\begin{equation}
 V=\frac{1}{2}\sum_{\vec l',\alpha'}\sum_{\vec l,\alpha}\vec u_{\vec l'\alpha'}\cdot\underbrace{\vec\Phi^{\alpha',\alpha}(\vec R_{\vec l}-\vec R_{\vec l'})}_\text{matriu de constants de forces}\cdot\vec u_{\vec l,\alpha}
\end{equation}

Les equacions de moviment en l'aproximació harmònica s'escriuen en la coneguda forma:

\begin{equation}
\label{eq:movharmonic}
 M_{\alpha'}\vec{\ddot{u}}_{\vec l,\alpha'}(t)=-\sum_{\vec l',\alpha}\vec\Phi^{\alpha',\alpha}\left(\vec R_\vec l-\vec R_{\vec l'}\right)\cdot\vec u_{\vec n,\alpha}(t)
\end{equation}
 Aquesta equació representa un sistema  d'oscil·ladors harmònics acoblats, on $\alpha$ i $\alpha'$ fan referència al àtoms de la base considerats, l'índex vectorial $\vec l$ (que sol aparèixer en la literatura com $n$) índica el nuc considerat i $\vec R_\vec l$ és el vector de translació cristal·lina.
 
\subsubsection{Matriu dinàmica}
Realitzant la transformada de Fourier de l'equació \ref{eq:movharmonic}, passem al conegut problema d'autovalors:

\begin{equation}
\sum_{\alpha}\vec D^{\alpha'\alpha}(\vec q)\cdot\vec e_{\alpha}(\vec q)={\omega'}^{2}\vec e_{\alpha'}(\vec q)
\end{equation}   

La matriu dinàmica és la magnitud central de la dinàmica reticular: les freqüències dels fonons es calculen a partir dels valors propis de la matriu dinàmica.

Per tant, les freqüències $\omega$ com funció del vector d'ones $\vec q$ del fonó són solució de l'equació secular:

\begin{equation}
\label{eq:secular}
\det\left|\frac{1}{\sqrt{M_{\alpha'} M_{\alpha}}}\vec D^{\alpha'\alpha}\left(\vec q\right)-{\omega'}^2\left(\vec q\right)\right|=0 
\end{equation}

on $M_{\alpha}$ es la massa de l'àtom $\alpha$ i la matriu dinàmica ve definida per:

\begin{equation}
D^{\alpha',\alpha}_{i,j}=\frac{\partial^2 E}{\partial u^{\alpha'*}_i(\vec q)\partial u^{\alpha}_{j}(\vec q)}
\label{eq:Matriz_Dinámica}
\end{equation}

on $u^{\alpha}_{i}$ representa el despla\c{c}ament de l'àtom $\alpha$ en la direcció $i$.

La segona derivada de l'energia de l'equació \ref{eq:Matriz_Dinámica} correspon al canvi en la for\c{c}a que actua sobre l'àtom $\alpha$ en la direcció $j$ quan es despla\c{c}a l'àtom $\alpha'$ en la direcció $i$

\begin{equation}
D^{\alpha',\alpha}_{i,j}=\frac{\partial}{\partial u^{\alpha*}_{i}}F^{j}_{\alpha}(\vec q)
\end{equation}


\begin{equation}
 \label{eq:matriu_dinàmica}
 \boxed{
 \vec D^{\alpha',\alpha}(\vec q)=\frac{1}{\sqrt{M_{\alpha'} M_{\alpha}}}\sum_\vec l\vec\Phi^{\alpha',\alpha}\left(\vec R_\vec l-\vec R_{\vec l'}\right)e^{-i\vec q\cdot\vec R_\vec l}}
\end{equation}


%Obtesses les posicions dels àtoms y classificats estos com primers, segons, etc. veïns, segons la distància al respectiu àtom de la ce\l.la $\vec 0$, procedim a calcular la contribució a la matriu dinàmica de cadascun dels àtoms, per la qual cosa necessitem conèixer el tensor de constants de for\c{c}a que correspon a la interacció de cada àtom amb el seu n-èssim veí.


Per calcular la matriu dinàmica emprant el model de constants de forces necessitem, per tant, construir el tensor de constants de forces. Anem a suposar que les forces entre àtoms sols depenen del tipus d'elements químics que interaccionen i de la seua distància relativa.

Considerem un àtom $\alpha$ situat en la cel·la $\vec l$ , ($\alpha, \vec l$),a una certa distància, $|\vec R_{\alpha,\vec l}-\vec R_{\alpha',\vec{l'}}|$ de l'àtom  ($\alpha'$,$\vec l'$), i escollint el sistema de coordenades cartesianes de manera que la direcció del vector que uneix ambdós àtoms coincideix amb l'eix de les $x$, l'eix $y$ com la coordenada transversal en el plànol, $ti$, i $z$ la coordenada perpendicular al plànol $to$.

Podem escriure el tensor de forces d'este àtom, que segons el mòdul de la distància a l'àtom $(\alpha',\vec l$) classificarem com $n$-èssim veí, de la forma \cite{wirtz04_phonon_disper_graph_revis}:


\begin{equation}
\vec\Phi_n^{\alpha',\alpha}=\begin{pmatrix}
\phi_{n,r}^{\alpha',\alpha}&\xi_n^{\alpha',\alpha} &0\\
-\xi_n^{\alpha',\alpha} & \phi_{n,ti}^{\alpha',\alpha} & 0 \\
0 & 0 & \phi_{n,to}^{\alpha',\alpha}
\end{pmatrix}
\label{eq:tensordeforces}
\end{equation}

L'estructura diagonal a blocs de la matriu reflexa el fet que estem suposant que  les vibracions interplanars i les de fora de pla, $to$, (en la direcció $z$) estan completament desacoblades.

Anem a suposar (simplificant encara més com són les interaccions entre àtoms) que un despla\c{c}ament longitudinal (radial, que estarà contés en el plànol del cristall) o transversal en el plànol, sols genera una força radial o transversal també, es a dir, $\xi_n^{\alpha,\alpha'}=0$ tal i com es realitza en \cite{Balkanski_2000}. %Esta aproximació es coneix com la aproximació dels quatre veïns més pròxims \textit{4NNFC},% i es necessita considerar fins els cuarts veïns per donar compte dels resultats experimentals.

%\missingfigure{Ací va imatge mostrant el cristall i les forces}


%\begin{figure}[h]
%\centering
%\includegraphics[width=40mm,height=40mm]{example-image-a}
%\caption{Caption}
%\end{figure}


%Una altra aproximació al problema trobada en la literatura sobre els fonons del grafé és la coneguda com el modelo \textit{VFF} (\textit{valence-force field}), la qual determina els paràmetros de la matriu en l'equació \ref{eq:tensordeforces} introduïnt \textit{constants de moll} que determinen el canvi en l'energia potencial segons diferents deformacions; una introducción a esta aproximació es troba en l'annexe de la referència (cite aizawa90 bond soften monol graph formed). Amb aquesta aproximació es necesiten menys paràmetres (\textit{constantes de fuerza}) per  obtindre una qualitat similar a la parametrizació \textit{4NNFC} (cite wirtz04 phonon disper graph revis).

Per tant, anem a considerar que el tensor de constants de forces  d'un àtom $\alpha$ classificat com $n$-èssim veí, situat en la direcció $\hat x$ respecte de l'àtom  $\alpha'$, té la forma diagonal (notem que en la xarxa real que estem estudiant no té perquè haver cap $n$-èssim veí en aquesta posició):

\begin{equation}
\vec\Phi_n^{\alpha'\alpha}=\begin{pmatrix}
\phi_{n,r}^{\alpha',\alpha}&0 &0\\
0& \phi_{n,ti}^{\alpha',\alpha} & 0 \\
0 & 0 & \phi_{n,to}^{\alpha',\alpha}
\end{pmatrix}
\label{eq:tensordeforcessimplificat}
\end{equation}

Per calcular el tensor de constants de forces de cadascun dels $i$ $n$-èssim veí real de l'àtom $\alpha'$, que formen un angle $\theta$ respecte de l'eix escollit com $x$, tenim que rotar la matriu de l'equació \ref{eq:tensordeforcessimplificat}:

\begin{equation}
 \vec\Phi_{n,i}^{\alpha'\alpha}(\theta)=\vec U^{-1}(\theta)\vec\Phi_n^{\alpha'\alpha}(0)\vec U(\theta)
\end{equation}


on $\vec U(\theta)$ ve donada per:
\begin{equation}
\vec U(\theta)=
\begin{pmatrix}
\cos(\theta)  & \sin(\theta) & 0 \\
-\sin(\theta) & \cos(\theta) & 0  \\
0             & 0            & 1
\end{pmatrix}
\end{equation}


Una vegada que sabem com tenim que construir el tensor de constants de forces, el càlcul de la matriu dinàmica és directe, ja que sols tenim que fer ús de l'equació \ref{eq:matriu_dinàmica}, és a dir multipliquem el tensor de constants de forces associat a la interacció entre els àtoms ($\alpha', \vec l'=\vec 0$) i ($\alpha,\vec l$) per la fase, $e^{i \vec q\cdot \vec R_\vec l}$, on recordem, $\vec R_\vec l=l_1 \vec a_1+ l_2 \vec a_2$

En el cas considerat, com la base esta constituïda per dos àtoms, la dimensió del tensor de constants de forces (i de la matriu dinàmica) és $3\cdot 2\times 3\cdot 2= 6\times 6$. Observem que esta matriu està escrita en termes de $4$ matrius $3\times 3$:

\begin{equation}
 \label{eq:formamatriudinamica}
 \mathbf D=
 \begin{pmatrix}
  \mathbf D^{BB} && \mathbf D^{BN}\\
  \mathbf D^{NB} && \mathbf D^{NN}
 \end{pmatrix}
\end{equation}

recalquem que hem indexat el àtoms de la base pel tipus d'element al que pertanyen, però en cas que els $2$ àtoms de la base pertanyeren al mateix tipus d'element químic la matriu dinàmica també seria $6x6$ (i podríem, per exemple, classificar-los com $A$ i $B$ per distingir-los).

Un punt important és que tenim que considerar (en les submatrius $D^{BB}$ i $D^{NN}$) les contribucions a la matriu dinàmica degudes a la interacció de l'àtom ($\alpha',0$) amb sí mateix. Ara bé, no necessitem escriure $\vec D_0^{BB}$ i $\vec D_0^{NN}$ explícitament, podem emprar el fet que si traslladem el conjunt d'àtoms en la mateixa direcció la força total és nul·la (si es desplaça el cristall com un tot no canvia l'energia potencial) i tal com ve indicat en \cite{falkovsky08_symmet_const_phonon_disper_graph} a partir de l'equació \ref{eq:matriu_dinàmica}, obtenim que (considerant fins a 3ers veïns):

%En el codi escrit per calcular l'expressió de la matriu dinàmica pot observar-se que la matriu dinàmica està construïda a capes (realitzant el càlcul sobre veïns situats a la mateixa distancia i sumant estes contribucions) i en les submatrius $3\times 3$ del mateix tipus d'elements, on tindríem que sumar la contribució de $\vec D^{AA}_0$, esta no l'he escrit explícitament, ja que es pot calcular a partir del fet que si traslladem el conjunt d'àtoms en la mateixa direcció la força total és nula (si es desplaça el cristall com un tot no canvia l'energia potencial) i tal com ve indicat en \cite{falkovsky08_symmet_const_phonon_disper_graph} a partir de l'equació \ref{eq:matriu_dinàmica}, obtenim que:

\begin{equation}\begin{split}
\vec D_0^{BB}&=-\sum_{i=1}^3\vec \Phi_{1,i}^{BN}-\sum_{i=1}^6\vec \Phi_{2,i}^{BB}-\sum_{i=1}^3\vec \Phi_{3,i}^{BN} \\
\vec D_0^{NN}&=-\sum_{i=1}^3\vec \Phi_{1,i}^{NB}-\sum_{i=1}^6\vec \Phi_{2,i}^{NN}-\sum_{i=1}^3\vec \Phi_{3,i}^{NB}
\end{split}
\end{equation}


Es podrien tindre en compte altres simetries del cristall per determinar certes propietats del tensor de forces o de la seua transformada de Fourier, la matriu dinàmica, (certes relacions entre les components, com es fa en \cite{falkovsky08_symmet_const_phonon_disper_graph}) però sol he emprat el fet que donat que l'energia es una funció quadràtica dels desplaçaments atòmics $\vec u_{\alpha',\vec l'}, \vec u_{\alpha,\vec l}$, la matriu de constants de forces té que ser simètrica,  i la seua transformada de Fourier, la matriu dinàmica té que ser una matriu hermítica, i per tant els seus valors propis, $\omega^2$ seran nombres reals.


Tenim per tant que la matriu dinàmica serà una matriu $6x6$ hermítica. Però, donat que les components en $z$ d'esta matriu es troben desacoblades, podem tractar aquestes vibracions de manera independent, de manera que en compte de treballar directament sobre una matriu $6x6$ de la forma

\begin{equation}
  \vec D(\vec q)=
  \begin{pmatrix}
   D_{xx}^{BB} & D_{xy}^{BB} & 0 &              & D_{xx}^{BN} & D_{xy}^{BN} & 0 \\
   D_{yx}^{BB} & D_{yy}^{BB} & 0 &              & D_{xx}^{BN} & D_{xy}^{BN} & 0 \\
       0       &     0       & D_{zz}^{BB} &    & 0  & 0  & D_{zz}^{BN}        \\
    &  &  &  &  &  \\
   D_{xx}^{NB} & D_{xy}^{NB} & 0 &              & D_{xx}^{NN} & D_{xy}^{NN} & 0 \\
   D_{yx}^{NB} & D_{yy}^{NB} & 0 &              & D_{xx}^{NN} & D_{xy}^{NN} & 0 \\
       0       &     0       & D_{zz}^{NB} &    & 0  & 0  & D_{zz}^{NN} 
  \end{pmatrix}
\end{equation}

treballarem sobre dues matrius 

\begin{enumerate}
 \item Primer tractarem les vibracions transversal al pla del cristall mitjançant una matriu $2x2$:
 \begin{equation}
  \vec D_z(\vec q)=
  \begin{pmatrix}
    D_{zz}^{BB}  & D_{zz}^{BN}        \\
    D_{zz}^{NB}  & D_{zz}^{NN} 
  \end{pmatrix}
\end{equation}

 \item Després tractarem les vibracions al pla del cristall emprant una matriu $4x4$:
 \begin{equation}
  \vec D_{xy}(\vec q)=
  \begin{pmatrix}
   D_{xx}^{BB} & D_{xy}^{BB} & D_{xx}^{BN} & D_{xy}^{BN}  \\
   D_{yx}^{BB} & D_{yy}^{BB} & D_{xx}^{BN} & D_{xy}^{BN}  \\
   D_{xx}^{NB} & D_{xy}^{NB} & D_{xx}^{NN} & D_{xy}^{NN} \\
   D_{yx}^{NB} & D_{yy}^{NB} & D_{xx}^{NN} & D_{xy}^{NN} \\
  \end{pmatrix}
\end{equation}

\end{enumerate}


 
\newpage


\section{Eines emprades}

Els càlculs s'han realitzat en el llenguatge de programació \href{https://www.python.org/}{python} \cite{4160250}, emprant el software matemàtic \href{https://www.sagemath.org/}{Sagemath} \cite{sagemath}. Sagemath està construït sobre python i facilita un accés unificat a biblioteques orientades a càlcul numèric com \href{https://numpy.org/}{numpy} \cite{harris2020array} , \href{https://scipy.org/}{scipy} \cite{2020SciPy-NMeth}, biblioteques orientades a càlcul simbòlic com \href{https://www.sympy.org/}{sympy} i el sistema d'àlgebra computacional \href{https://maxima.sourceforge.io/}{Maxima}, entre molts altres. Si es necessita emprar python o qualsevol de les seues biblioteques es poden emprar directament també; en este treball, per exemple, he emprat també la biblioteca \href{https://pandas.pydata.org/}{pandas}\cite{reback2020pandas} 

Com a entorn de desenvolupament s'ha emprat \href{https://jupyter.org/}{jupyterlab} i \href{https://git-scm.com/}{git} com sistema de control de versions distribuït. 

El projecte, incloent els \textit{\foreignlanguage{english}{jupyter notebooks}} amb els càlculs, es troba accessible al repositori públic de github \href{https://github.com/CasimirVictoria/TFG-Semiconductores\_2D}{TFG-Semiconductores\_2D}. Així mateix, una versió en línia, orientada a mostrar els càlculs realitzats, es troba publicada en \href{https://casimirvictoria.github.io/TFG-Semiconductores_2D/index.html}{github pages}; la creació i publicació d'esta versió online s'ha realitzat automàticament a partir dels \textit{Jupyter notebooks} emprant \href{https://jupyterbook.org/intro.html}{jupyterbook}.

Jupyter/Jupyterlab ha demostrat ser un entorn fantàstic (i molt popular) per a programar en python (i altres llenguatges), però al no treballar sobre fitxers de text pla (en este cas fitxers amb codi python), dificulta emprar eines de control de versions de codi com git. Per aquesta raó s’ha emprat també el plugin \href{https://jupytext.readthedocs.io/en/latest/}{jupytext} per a Jupyter, que permet tindre sincronitzat el notebook amb un fitxer en text pla (amb el codi en python, markdown, etc.). En este projecte he decidit emprar el format \href{https://jupyterbook.org/content/myst.html}{myst markdown}, de manera que des de jupyterlab òbric un fitxer markdown i la extensió jupytext sincronitza un fitxer en format \textit{.ipynb} amb este fitxer, que és amb el qual realment treballa jupyterlab. Este \emph{notebook} l'exporte després a un fitxer python que carregue posteriorment al fitxer .tex on he escrit la memòria, emprant el paquet \href{https://ctan.org/pkg/sagetex}{sagetex}, de manera que puc incloure tots els càlculs realitzats (incloent expressions analítiques, no sols numèriques i gràfics) de manera automàtica en la memòria sense tindre que escriure-les manualment, aprofitant l'excel·lent suport que té sagemath de \LaTeX . 

Al repositori de github, dins de la carpeta \textit{Memoria} es troba un petit \textit{script} en \textit{bash} que automatitza la generació d'esta memòria si realitze algun canvi sobre el notebook amb els càlculs.

Notem que tot els software emprat per a la realització del treball és de codi obert i gratuït.


\newpage
\section{Descripció del cristall de BN}

Passem ara a descriure el sistema que s'ha estudiat i els resultats dels càlculs realitzats.

Donat que el càlcul dels modes de vibració comença per establir la geometria del cristall en equilibri, comprovem amb les dades proporcionades que el $BN$ monocapa es tracta d'un cristall bidimensional hexagonal de base diatòmica, la cel·la unitat del qual ve donada per (dades proporcionades):

\begin{sagesilent}
load("formulacio_matriu_dinamica.sage")
\end{sagesilent}

\begin{equation}
\vec a_1=\sage{a_1}\qquad\vec a_2=\sage{a_2} 
\end{equation}

Podem comprovar que efectivament es tracta de una ce\l.la hexagonal, ja que els seus vectors primitius formen un angle de $\sage{arccos(a_1*a_2/(norm(a_1)*norm(a_2)).simplify())}$ \SI{}{\radian}.

Les posicions atòmiques d'equilibri en la ce\l.la unitat són:

\begin{equation}\begin{split}
\vec d_B=&\frac{1}{3}\vec{a_1}+2\vec{a_2}=\sage{r_B}\qquad\\
\vec d_N=&\frac{2}{3}\vec{a_1}+\frac{1}{3}\vec{a_2}=\sage{r_N} 
\end{split}\end{equation} 



\begin{figure}[h]
\centering
\begin{subfigure}[b]{0.3\textwidth}
\centering
\sageplot[width=\textwidth]{xarxa, figsize=3}
\caption{Xarxa de nucs}
\end{subfigure}
\begin{subfigure}[b]{0.3\textwidth}
\centering
\sageplot[width=\textwidth]{cela, figsize=3}
\caption{Ce\l.la unitat}
\end{subfigure}
\begin{subfigure}[b]{0.3\textwidth}
\centering
\sageplot[width=\textwidth]{AtomsB+AtomsN, figsize=3}
\caption{Cristall}
\end{subfigure}
\end{figure}


\paragraph{Xarxa $2\pi$-recíproca}
Passe ara a mostrar la xarxa $2\pi$-recíproca associada a la xarxa de nucs del cristall de $BN$, i calcular els punts de màxima simetria de la primera zona de Brillouin:

Els vectors primitius de la xarxa $2\pi\text{-recíproca}$ són:

\begin{equation}
\label{eq:11}
\vec b_1=\frac{2\pi}{a}\left(\hat k_{x}+\frac{1}{\sqrt{3}}\hat k_{y}\right)\quad \vec b_2=\frac{2\pi}{a}\left(\frac{2}{\sqrt{3}}\hat k_y\right)
\end{equation} 

per tant, la xarxa $2\pi$-recíproca també és una xarxa hexagonal (els vectors base formen un angle de $\sage{arccos(b_1*b_2/(norm(b_1)*norm(b_2)).simplify())}$ \SI{}{\radian}

Els punts crítics de la zona de Brillouin són:

\begin{equation}
 \label{puntsbrillouin}
 \Gamma=(0,0)\quad M=\left(q_x=\pi/a,q_y=\frac{\pi}{\sqrt{3 a}}\right)\quad K=\left(q_x=\frac{4\pi}{3 a}, q_y=0\right)
\end{equation}

\newpage
\begin{figure}[h]
\centering
\sageplot[width=6cm]{xarxaR, figsize=4}
\caption{Xarxa $2\pi$-recíproca}
\end{figure}


Passem ja a estudiar les vibracions en el $BN$.
\subsection{Vibracions transversals fora de pla}
Donat que en el nostre model, per com hem construït la matriu dinàmica, les vibracions fora de pla són independents de les interplanars passem a estudiar les primeres, ja que són més senzilles (treballarem amb una matriu 2x2). Considere sols fins a tercers veïns.

En este cas es poden realitzar els càlculs analíticament.

\subsubsection{En $\Gamma$}
Al punt $\Gamma$, ($q_x=0, q_y=0$), la matriu dinàmica  per a les vibracions fora de pla pren el valor:

\begin{equation}
\vec D_z(\Gamma)=\sage{D_Gamma3ers_zz}
\end{equation}

i es seus valors propis son:

\begin{equation}\begin{split}
\omega_{ZO}^2(\Gamma)&=\sage{D_Gamma3ers_zz.eigenvalues()[0]}\\
\omega_{ZA}^2(\Gamma)&=\sage{D_Gamma3ers_zz.eigenvalues()[1]} 
\end{split}
\end{equation}



on el valor no nul en $\vec q=\vec 0$ correspon a una rama òptica, denominada $ZO$ per fet que es transversal fora de pla (moviment en l'eix $z$) i òptica. La freqüència nul·la correspon a un mode acústic (recordem en este sentit que les rames acústiques es caracteritzen per la seua freqüència nul·la quan en $\vec q=\vec 0$). 

Podem observar que coincideixen, formalment, amb els que trobem en \cite{falkovsky08_symmet_const_phonon_disper_graph} per al grafè.

\subsubsection{En M}

Al punt $M$ $\left(q_x=\pi/a,q_y=\pi/(\sqrt 3 a)\right)$

Els valors propis de la matriu:

\begin{equation}
\vec D_z(M)=\sage{D_M3ers_zz}
\end{equation}

són:

\begin{equation}\footnotesize{
 \begin{split}
\omega_{ZO}^2(M)&={\sage{omegaM2quadrat3ers}}\\
\omega_{ZA}^2(M)&=\sage{omegaM1quadrat3ers}
 \end{split}}
\end{equation}

on el valor més baix correspon a la rama \emph{ZA} i la freqüència més alta a la rama òptica $ZO$.
Podem comprovar que si considerem que els àtoms foren del mateix tipus obtenim exactament les mateixes expressions que en \cite{falkovsky08_symmet_const_phonon_disper_graph}

\subsubsection{En K}

Al punt $K$ $(q_x=4\pi/(3 a)$, $q_y=0)$

Els valors propis de la matriu $ \vec D_z(K) $ són (no es mostra l'expressió de la matriu en esta memòria, si es vol consultar es troba en la \href{https://casimirvictoria.github.io/TFG-Semiconductores_2D/index.html}{versió en línia del notebook amb els càlculs}):

\begin{equation}\footnotesize{
 \begin{split}
\sage{Eq_K_ZA3ers}\\
\sage{Eq_K_ZO3ers}
 \end{split}}
\end{equation}

Observem també que les equacions per als modes al punt $K$ són formalment idèntiques a les que trobem en \cite{falkovsky08_symmet_const_phonon_disper_graph} per al grafè, sols que en aquest cas, com estem tractant amb una base amb dos àtoms des distint tipus no tenim degeneració en este punt (el segon terme d'ambdues equacions es diferent en este cas).
Tenim que tindre en compte que en no sabem, a priori, a quina rama correspon cada valor propi, segons l'elecció que realitzem obtindrem els valors de les constants $\phi_{2,to}^{NN}$ i $\phi_{2,to}^{BB}$ intercanviades.

\paragraph{Si considerem fins els quarts veïns}
Si en compte de fins als tercers veïns tenim en compte fins els quarts, les expressions que obtinc són (no es mostren les expressions per al punt $M$ ja que són massa llargues, però es poden consultar en la versió en línia):

\begin{equation}\footnotesize{
 \begin{split}
 \sage{Eq_Gamma_ZO4ts}\\
 \sage{Eq_K_ZO4ts}\\
 \sage{Eq_K_ZA4ts}
 \end{split}}
\end{equation}

\subsubsection{Determinació de les constants de força}

Passem a resoldre el sistema d'equacions que hem obtès per determinar les constants de força (considerant fins als 3ers veïns).
Tenim més equacions que incògnites, i segons les equacions escollides obtenim valors diferents. El sistema d'equacions que obtenim el podem resoldre, en este cas, tant analítica com numèricament.
En la \href{https://casimirvictoria.github.io/TFG-Semiconductores_2D/index.html}{versió en línia del notebook amb els càlculs} es mostren diferents solucions trobades (tan analítiques com numèriques), ací es mostra la solució que millor ajusta a les corbes de dispersió proporcionades.

Les constants de forces s'han obtès minimitzant:

\begin{equation}
\sqrt{\sum_{n=1}^3\sum_{i=1}^2\left((\omega_{i,n}^\text{model})^2-(\omega_{i, n}^\text{DFPT})^2\right)}
\end{equation}


\begin{equation}
 \sage{Solucions3ers}
\end{equation}

Les dues corbes de dispersió que obtenim amb aquesta solució es mostren a continuació (en roig), front a les corbes de dispersió calculades mitjançant el mètode de \emph{Density Functional Perturbation Theory}

\begin{center}
\includegraphics[width=10cm]{../Grafiques/DisZ3ers.pdf}
\end{center}

\paragraph{Si considerem fins els quarts veïns}

Aconseguim obtindre els autovalors de la matriu dinàmica analíticament, però el sistema d'equacions que obtenim no puc resoldre'l algebraicament (emprant \textit{sagemath}, que al seu torn permet emprar tant la biblioteca \textit{sympy} com \textit{maxima} per realitzar els càlculs).  Resolguent el problema numèricament obtenim un millor ajust a estes dues corbes (els càlculs es poden consultar en línia):

\begin{equation}\small{
 \sage{Solucions4ts}}
\end{equation}

\begin{center}
\includegraphics[width=10cm]{../Grafiques/DisZ4ts.pdf}
\end{center}

Passem ara a estudiar les vibracions dins del pla del cristall.

\subsection{Vibracions dins del pla del cristall}

En este cas tractem amb una matriu $4\times 4$, i per tant tenim 4 valors propis ($\omega^2$) en cada punt $\vec q$ de la primera zona de Brillouin.


\subsubsection{Punt $\Gamma$}

Al punt $\Gamma$ obtinc 2 valors propis, de multiplicitat $2$ cadascun, els que tenen valor propi $0$ corresponen a les dues rames acústiques (recordem que en un cristall bidimensional tenim $2(N_\alpha -1)$ rames òptiques (2 en el nostre cas ja que al tractar-se d'un cristall amb base diatòmica $N_\alpha=2$, i les altres $2$ són acústiques, per tant: 

\begin{equation}
 \begin{split}
\sage{Eq_Gamma_LO3ers}\\
\sage{Eq_Gamma_TO3ers}
 \end{split}
\end{equation}

\subsubsection{Als punts $M$ i $K$}

Per als punts $M$ i $K$, sagemath (tant mitjançant \href{https://maxima.sourceforge.io/}{maxima} com \textit{sympy}) no obté una expressió analítica, però si abans simplifique el problema suposant que els àtoms de la base son iguals:

\begin{itemize}

\item Els dos àtoms tenen la mateixa massa, $M_N=M_B$
\item  $\phi_{2,r}^{NN}=\phi_{2,r}^{BB}$, $\phi_{2,ti}^{NN}=\phi_{2,ti}^{BB}$ 
\end{itemize}

sí que obté una solució analítica. Després obtinc valors propis correctes per al nostre model realitzant les substitucions pertinents al resultat obtés: cada constant apareix sempre dividida per l'arrel quadrada de les masses pertinents.
Podem estar segurs que les expressions analítiques trobades són  realment els autovalors de la matriu $\vec D_{xy}$, ja que comprovem que es compleix l'equació \ref{eq:secular}


Sols mostre dues de les solucions trobades, al punt $K$, perquè les altres tenen expressions massa llargues

\begin{equation}\small{
 \begin{split}
\sage{Eq_K_TO3ers}\\
\sage{Eq_K_LA3ers}
 \end{split}}
\end{equation}

Podem observar que els valors propis trobats en $\Gamma$ i estos dos en $K$ són també idèntics als que apareixen en les equacions (10) de \cite{falkovsky08_symmet_const_phonon_disper_graph}, sols que en este cas al tindre dos àtoms distints en la base no hi ha degeneració per als valors propis mostrats en $K$ (i que correspondrien a $\omega_{1,2}^\text{in-pl}(K)$ de l'esmentada referència. Podem observar en la relació de dispersió de dita referència que el valor degenerat en $K$ correspon a les branques TO i LA, i ens serveix per assignar tals valors en este treball. Com en el cas de $K$ per a les vibracions fora de pla, no sabem a quina rama correspon a priori (prove i segons isca l'ajust escollisc). Les equacions per als valors propis $\omega_{3,4}^\text{in-pl}(K)$ de dit conjunt d'equacions de \cite{falkovsky08_symmet_const_phonon_disper_graph} podem vore que també es corresponen a les que hem calculat en este treball, ja que si bé les que obtenim ací són més complicades, si sumem ambdós valors propis al quadrat, com estos difereixen, igual que en \cite{falkovsky08_symmet_const_phonon_disper_graph} en que un dels termes de l'equació apareix sumant en un cas i restant en l'altre l'equació que obtenim:

\begin{equation}\small{\begin{split}
\label{sumaktaiklo3ers}
 &\sage{EqKTAmesEqKLO_3ers.lhs()}\\
 &=\sage{EqKTAmesEqKLO_3ers.rhs()}
 \end{split}}
\end{equation}

és formalment idèntica a la de dita referencia.

\section{Resultats}

Finalment, assignant a cadascuna de les equacions obtesses per als valors propis el valor numèric corresponent, obtenim el següent ajust:

\begin{center}
\includegraphics[width=10cm]{../Grafiques/Dispersio3ers.pdf}
\end{center}

Ampliant lleugerament el codi emprat per considerar fins el quarts veïns millorem un poc l'ajust:

\begin{center}
\begin{tabular}{|c|c||c|c|}\hline
$\sage{Solucions4ts[0].lhs()}$ & $\sage{Solucions4ts[0].rhs()}$ & $\sage{Solucionspla4ts[0].lhs()}$ & $\sage{Solucionspla4ts[0].rhs()}$  \\ \hline
$\sage{Solucions4ts[1].lhs()}$ & $\sage{Solucions4ts[1].rhs()}$ & $\sage{Solucionspla4ts[1].lhs()}$ & $\sage{Solucionspla4ts[1].rhs()}$  \\ \hline
$\sage{Solucions4ts[2].lhs()}$ & $\sage{Solucions4ts[2].rhs()}$ & $\sage{Solucionspla4ts[2].lhs()}$ & $\sage{Solucionspla4ts[2].rhs()}$  \\ \hline
$\sage{Solucions4ts[3].lhs()}$ & $\sage{Solucions4ts[3].rhs()}$ & $\sage{Solucionspla4ts[3].lhs()}$ & $\sage{Solucionspla4ts[3].rhs()}$  \\ \hline
$\sage{Solucions4ts[4].lhs()}$ & $\sage{Solucions4ts[4].rhs()}$ & $\sage{Solucionspla4ts[4].lhs()}$ & $\sage{Solucionspla4ts[4].rhs()}$  \\ \hline
  &  & $\sage{Solucionspla4ts[5].lhs()}$ & $\sage{Solucionspla4ts[5].rhs()}$  \\ \hline
  &  & $\sage{Solucionspla4ts[6].lhs()}$ & $\sage{Solucionspla4ts[6].rhs()}$  \\ \hline
  &  & $\sage{Solucionspla4ts[7].lhs()}$ & $\sage{Solucionspla4ts[7].rhs()}$  \\ \hline
  &  & $\sage{Solucionspla4ts[8].lhs()}$ & $\sage{Solucionspla4ts[8].rhs()}$  \\ \hline
  &  & $\sage{Solucionspla4ts[9].lhs()}$ & $\sage{Solucionspla4ts[9].rhs()}$  \\ \hline
\end{tabular}
\end{center}

\begin{center}
\includegraphics[width=14cm]{../Grafiques/Dispersio4ts.pdf}
\end{center}


Podem observar que el model  proporciona un bon ajust per a les branques fora de pla, ZO i ZA, així com per a les acústiques en el pla, sobretot la TA. Per a la branca LA, sobretot en la part central, a altes freqüències, els resultats no són bons, però la corba té la forma pertinent (encara que les freqüències estan infravalorades).

A altes freqüències, per a les branques TO i sobretot LO el model no ajusta bé, el TO podem considerar que el model el descriu qualitativament.


Substituïm aquestes constants en la matriu $\vec D(\vec q)$ $6x6$ calcule els autovalors i autovectors per a 524 punts en que he discretitzat el recorregut per la primera zona de Brillouin. Tenim 6 autovalors i 6 autovectors en cadascun d'estos punts, de 6 components cada autovector (les 3 primeres son les components cartesianes del vector desplaçament per l'àtom de bor i les 3 últimes del de nitrogen). El resultat es guarda en un fitxer \attachfile[author=Casimir]{../Resultats/matdyn_casi_4ts.modes} amb el qual podem, emprant \href{https://henriquemiranda.github.io/phononwebsite/index.html}{TSS Physics - Home}, visualitzar interactivament els modes de propagació al cristall de BN: polsant sobre qualsevol punt del diagrama de dispersió obtenim una simulació del moviment del àtoms del cristall:

\begin{center}
 \includegraphics[width=15cm]{../Resultats/Captura_animacio.png}
\end{center}

Està pàgina és realment útil per simular les vibracions cristal·lines.

\vspace{2cm}

Nota: en este pdf s'inclou el codi font en \LaTeX  del document:\attachfile{TFG-Casimir.tex}, així com el fitxer amb els autovalors i autovectors calculats \attachfile{../Resultats/matdyn_casi_4ts.modes}, i el fitxer generat a partir d'este últim per poder visualitzar  interactivament els modes de vibració del cristall \attachfile{../Resultats/BN.json}

\section{Conclusions / \emph{Conclusions}}

\begin{itemize}

\item Observant els resultats que s'han obtès es pot concloure, que si bé el model de constants de forces és un model realment senzill i amb limitacions evidents, ja que podem observar que a altes freqüències el model és incapaç de reproduir la relació de dispersió calculada per primers principis emprant el mètode de \emph {Density Functional Perturbation Theory}, este model sí és capaç de proporcionar una bona descripció de la relació de dispersió a baixes freqüències així com una descripció almenys qualitativa a altes freqüències (excepte per al mode LO on falla totalment) 



\item Hem de tindre en compte que hem escollit el model més senzill possible, amb un tensor de constants de forces diagonal, i també que per realitzar l'ajust a les dades proporcionades sols s'han emprat $3$ punts dels més de $524$ disponibles (si bé són els punts crítics de la zona de Brillouin).
\item Podem concloure, per tant, que este model de constants de forces és útils per obtindre un ajust ràpid si disposem de dades experimentals, i que amb quasi total seguretat pot millorar-se considerant més punts, per a la qual cosa no procediria obtenint les expressions analítiques als punts crítics com he fet ací, directament faria ús de numpy i scipy per realitzar l'optimització dels valors propis a les dades proporcionades).
\item També és evident que quants més veïns considerem, millor ajust proporciona el model. Com podem comprovar en este treball l'ajust millora al quan considerem els quarts veïns.
\end{itemize}

\vspace{\baselineskip}

\begin{otherlanguage}{english}\itshape

\noindent \begin{itemize}

\item Observing the obtained results we can conclude that the force-constants model is a really simple model with evident limitations: we can observe that at high frequencies the model is unable to reproduce the dispersion relation calculated by first principles using the \emph{Density Functional Perturbation Theory} method, this model is capable of providing a good description of the  dispersion relation at low freqüències and a qualitative description at high freqüències (except for the mode LO where totally fails)

\item We must keep in mind that we have chosen the simplest possible model, with a diagonal force-constant tensor, and also that to make the adjustment to the data provided  only $3$ points have been used out of the more than  $524$ available (although they are the critical points of the Brillouin area).

\item We can conclude, therefore, that this force-constants model is useful to obtain a quick adjustment if we have experimental data, and that with almost total certainty it can be improved considering more points, for which it I would not proceed calculating the analytical expressions at the critical points as I did here, I would directly make use of numpy and scipy to perform the optimization of the experimental eigenvalues.


\item It is also evident that the more neighbors we consider, the better fit the model provides. As we can see in this work the adjustment improves when we consider the fourth neighbors.
\end {itemize}

\end{otherlanguage}


\bibliography{TFG-Casimir}
\bibliographystyle{plain}

\end{document}
