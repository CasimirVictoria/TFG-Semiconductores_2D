\documentclass[12pt]{article} %,twoside,a4paper

\usepackage[a4paper,margin=24.5mm]{geometry} %\usepackage[left=2cm,right=2cm,top=1.5cm, bottom=2cm]

\usepackage[utf8]{inputenc}
\usepackage[english,catalan]{babel}
%%%\usepackage[english,spanish]{babel}
\usepackage{natbib}
\usepackage{float} 
\usepackage{graphicx}


% Establim el format de la pàgina:
%\usepackage{fancyhdr}

%\setlength{\parskip}{0.2cm}
%\usepackage[T1]{fontenc}
%\usepackage{newunicodechar}
%\newunicodechar{Ŀ}{\L.}
%\newunicodechar{ŀ}{\l.}
\usepackage{siunitx}
\usepackage{amsmath}
%\usepackage{adjustbox}
%\usepackage{tabularx}
\usepackage{mathtools}
\usepackage{color}
\usepackage{caption}
\usepackage{subcaption}
\usepackage[colorlinks=true,urlcolor=blue,linkcolor=blue,citecolor=blue]{hyperref}
%\usepackage{booktabs} %Publication quality tables in LaTeX.
%\usepackage{pdflscape}
%\usepackage{attachfile}
\usepackage{todonotes}
%\usepackage{pdfpages}

\usepackage{sagetex}
%\definecolor{gris}{RGB}{220,220,220}
\let\vec\mathbf %vectors en negreta en compte de arrow

\author{Casimiro Victoria Castillo}


\begin{document}

\newgeometry{margin=1in}
\begin{titlepage}
   \noindent\raisebox{0pt}[0pt][0pt]{
	\includegraphics[width=10.5cm]{./portada-TFG-LaTeX/marca-Facultat-Fisica-UV-1-linia.pdf}}\par
   \vspace{8.5cm}
   % Titles
   {\centering
      \scalebox{1.125}{\bfseries\sffamily\Huge
		Treball de Fi de Grau en Física}\par
      \rule{16.13cm}{1.5pt}\par
      \vspace{4.5cm}
      {\bfseries\sffamily\LARGE Fonons i espectroscòpia Ramman en semiconductors bidimensionals}\par
   }
   \vfill
   {\raggedleft\sffamily
		AGOST de 2021\par
      \vspace{\baselineskip}
		Alumne/a: Casimiro Victoria Castillo\par
      \vspace{\baselineskip}
		Tutor/a (1): Alberto García Cristóbal\par
		Tutor/a (2): Alejandro Molina Sanchez\par
   }
\end{titlepage}
\restoregeometry

%%%%%%%%%%%%%%%%%%%%%%%%%%%%%%
%%% Abstract
%%%%%%%%%%%%%%%%%%%%%%%%%%%%%%

%\begin{abstract}
\subsection*{Resum / \emph{Abstract}}

Els materials bidimensionals ($2D$) com el grafè són de gran interès tant per les seues propietats físiques exclusives com per les seues aplicacions potencials. L'estudi de la dinàmica de la xarxa cristal·lina (\textit{fonons}) d'estos materials és un requisit previ per entendre la seua estabilitat estructural i propietats tèrmiques, així com les seues propietats de transport i òptiques.   


Este Treball de Fi de Grau consisteix en la computació dels modes vibracionals de materials semiconductors 2D y la seua correlació amb els observables rellevants per a la interpretació dels experiments de dispersió de la llum.
\vspace{\baselineskip}

\begin{otherlanguage}{english}\itshape

\noindent Abstract\ldots
The long-range character of the dynamical matrix limits the validity of
force-constant approaches that take only interaction with few neighboring atoms into account. However, by fitting the force-
constants to the ab initio dispersion relation, we show that the popular 4th-nearest-neighbor force-constant approach yields an
excellent fit for the low frequency modes and a moderately good fit (with a 
maximum deviation of $6 \%$) for the high-frequency modes \cite{wirtz04_phonon_disper_graph_revis}
\end{otherlanguage}

%\textit{Nota: s'ajunta el codi font del document: \attachfile[author=Casimir]{TFG-Casimir.tex}}

%\newpage
%\end{abstract}

\section{Introducció}

En este treball s'empra el mètode de constants de força (\emph{FCM} per les seues sigles en anglés, \foreignlanguage{english}{force constant method}), per descriure la dinàmica reticular de un material bidimensional, el nitrur de bor monocapa, \emph{BN}.\\

En este model la dinàmica dels àtoms es descriu per un conjunt de simples constants de força de molls que conecten cada àtom amb els del seu entorn, fins a un nombre determinat de veïns. Altres métodes com els càlculs realitzats a partir de primers principis (\emph{ab initio}) emprant la teoría del funcional densitat (DFT, per les seues sigles en anglés \foreignlanguage{english}{density-functional density}) o el mètode de camp de força de valència (\emph{VFF}, \foreignlanguage{english}{valence force field}) necessiten temps de càlcul molt més llargs. El métode de constants de força empra un conjunt reduït de parámetres que poden ajustar-se a les mesures experimentals o bé, com s'ha fet en este treball, a la relació de dispersió calculada per primers principis.  I encara que estem tractant amb un mètode sencill, este ha demostat que pot proporcionar resultats fiables \cite{wirtz04_phonon_disper_graph_revis}.


\subsection{Xarxa cristal·lina}

Abans d'explicar el mètode emprat en este treball repassem alguns conceptes i aclarim la notació emprada.

Sabem que els àtoms que constitueixen un sòlid cristal·lí estan distribuïts de una manera regular en l'espai, y per descriure esta distribució regular s'empra el concepte matemàtic de \emph{xarxa cristal·lina}, introduït per A. Bravais en 1845: una xarxa cristal·lina es defineix com una distribució discreta i regular de punts que té sempre la mateixa aparença independentment de quin punt escollim com origen de coordenades.

En el cas que anem a tractar, un cristall bidimensional, la xarxa cristal·lina bidimensional pot generar-se a partir de dos vectors base, de manera que els vectors de posició dels \emph{nucs de la xarxa}, els vectors de translació cristal·lina,  tenen la forma:
\begin{equation}
 \vec R_\vec l=\vec R_{l_1 l_2}=l_1 \vec a_1 +l_2\vec a_2
\end{equation}

on $\vec l$ es denomina índex del nuc.

Els cristalls o \emph{estructures cristallines} són distribucions regulars d'àtoms en l'espai de posicions i els podem descriure associant a cadascun dels nucs d'una xarxa cristallina en l'espai de posicions un conjunt d'àtoms o base (\emph{producte de convolució de la base y la xarxa de nucs}. En el cas del \emph{BN} tractem amb una base diatòmica. 

\subsubsection{Xarxa recíproca. Primera zona de Brillouin.}
Donat un conjunt de vectors base primitius, $p_i$, de la xarxa cristalina en l'espai de posicions, la condició:

\begin{equation}
\label{eq:rec1}
\vec p_i\cdot\vec p_j^{*}=\delta_{ij}
\end{equation}

on $\delta_{ij}$ és la delta de Kronecker, defineix un altre conjunt de vector $p_j^*$, coneguts com vectors primitius recíprocs, i què son els vectors primitius que defineixen una altra xarxa coneguda com coneguda com \textit{xarxa recíproca}. Els  vectors recíprocs, i els vectors de translació cristal·lina recíprocs, tenen dimensions de inversa de longitud i es representen en l'\textit{espai recíproc} o de números d'ona. \textbf{Les xarxes cristal·lines real y recíproca són dues descripcions equivalents del mateix sistema físic: el sòlid cristal·lí} que s'està estudiant.

L'espai $2\pi$-recíproc és el fonamental en l'estudi dels sòlids cristal·lins, ja que els estats de les partícules i les interaccions físiques de interés es descriuen en l'espai de vectors d'ona, $\vec k$.



Podem interpretar que 

\begin{equation}
\label{eq:rec3}
\vec K_{h_1h_2}=2\pi\left(h_1\vec a_1^{*}+h_2\vec a_2^{*}\right)
\end{equation}

són els vectors de traslació cristal·lina que defineixen una xarxa cristal·lina en aquest espai $2\pi$-recíproc (sóls es diferencia de l'espai recíproc en un factor d'escala $2\pi$). En térmes físics, l'espai $2\pi$-recíproc és l'\emph{espai de vectors d'ona} $\vec k$, i a falta d'un factors d'escala $\hbar$, coincideix amb l'espai de moments  $\vec p=\hbar\vec k$. 

\paragraph{Cel·les de Wigner-Seitz (WS) y primera zona de Brillouin (ZB)}

Per descriure la xarxa $2\pi$-recíproca, emprarem el criteri de \textit{Wigner-Seitz}. Las \textit{cel·les de Wigner-Seitz} estan centrades en un nuc de la xarxa i es defineixen com la regió més pròxima a un nuc (el del centre de la cel·la) que a qualsevol altre. Per determinar la seua forma, partim d'un nuc qualsevol com a origen, construïm els segments que uneixen este nuc amb els seus veïns i es tracen els plans que bisecten cadascun d'estos segments: la cel·la de WS és la cel·la de menor volum al voltant de l'origen que està delimitada per estos plans (rectes en el caso d'una xarxa bidimensional).

Notem que en l'espai de $\vec k$ emprarem ce\l.les unitat de WS mentre que en l'espai de posicions sempre emprem ce\l.les unitat de Bravais.
La ce\l.la de WS de la xarxa $2\pi$-recíproca es coneix com \textbf{primera zona de Brillouin} (ZB)


\subsection{Vibracions atòmiques en cristalls}
\subsubsection{Model de BORN i VON KARMAN}
Els àtoms constituents de un sólid cristallí no estan immòbils sino que vibren al voltant de la seua posició de equilibri.

En 1912, Born y Von Karman \cite{Born:1912:SRG} introduiren un model realístic per de la dinámica cristal·lina, on la idea fonamental es que cada mode normal té l'energia de un oscil·lador de Planck. La dinámica del sistema es descriu de manera sencilla, no en termes de les vibracions de àtoms individuals, sino en termes de moviments col·lectius en forma de ones viatgeres anomenades vibracions cristal·lines (\foreignlanguage{english}{lattice vibrations}) per Born. Cada vibració cristal·lina es caracteritza per una freqüencia y un vector de ona.
La vibració cristal·lina quantitzada, o mode normal, s'anomena fonó per analogía amb el fotó. %i te propietats de quasi-partícula

%Explicar el concepte de fonó i la seua importància: Un fonó és un mode de vibració quantitzat que té lloc en una xarxa cristal·lina. Els fonons tenen una gran importància en moltes propietats físiques dels sòlids. Los fonons son una version mecano-quàntica dels modes normals de vibració de la mecànica clàssica, on cada par de la xarxa osci\l.la amb la mateixa freqüència. Aquests modes normals són importants perquè qualsevol moviment vibracional de la xarxa pot descriure's com una superposició de modes normals de distinta freqüència, en este sentit són la base de les vibracions de la xarxa.

\subsubsection{Aproximació adiabàtica}

Moltes propietats físiques dels sòlids poden classificar-se bé com electròniques o bé com vibracionals segons estiguen determinades pels electrons (de valència) o per les vibracions dels àtoms (nuclis iònics): normalment considerem els nuclis i els electrons com a constituent independents del sòlid, ja que les masses dels electrons i dels nuclis són tan diferents que el moviment dels nuclis és molt més lent que el dels electrons. L'\emph{aproximació adiabàtica}, introduïda per Born i Oppenheimer \cite{ANDP:ANDP19273892002} estableix que els electrons responen de manera pràcticament instantània als desplaçaments atòmics, de manera que el seu estat ve determinat per los posicions atómiques instantànies, mentre que els àtoms no poden respondre a les ràpides variacions espacials dels electrons: diguem que els electrons segueixen el moviment iònic adiabàticament.
%els electrons no realitzen transicions abruptes de un estat a altres estats, si no que un estat electrònic es deforma progressivament degut als desplaçaments iònics.

Notem però que estem tractant amb una primera aproximació al problema y algunes propietats físiques venen determinades per la interacció entre els electrons y les vibracions atòmiques.



%podem escriure l'energia potencial (o el Hamiltonià) como una suma de les distintes contribucions.
  
\subsubsection{Aproximació harmònica}

Les vibracions reticulars estan regides per les forces que experimenten els àtoms quan es desplacen de la seua posició d'equilibri. La primera hipòtesi és que cada àtom té una posició d'equilibri en el cristall, que denotem per $\vec r^{(0)}_{\vec l,\vec\alpha}=\vec R_\vec l+\vec d_\alpha$ y considerarem que estos àtoms vibren al voltant d'esta posició d'equilibri, $\vec r_{\vec l,\vec\alpha}=\vec r^{(0)}_{\vec l,\vec\alpha}+\vec u_{\vec l,\alpha}(t)$, amb una amplitud menuda (en comparació amb la distància interatòmica) al voltant d'aquesta,   de manera que el sòlid es troba en estats que corresponen al que macroscòpicament es coneix com \textit{la regió de comportament elàstic lineal}, on es verifica la llei de Hooke.

Podem per tant aproximar l'energia potencial de interacció pel terme harmònic del seu desenvolupament en sèrie de potencies del despla\c{c}ament:

Les equacions de moviment en l'aproximació harmònica s'escriuen en la coneguda forma:

\begin{equation}
 M_\alpha\vec{\ddot{u}}_{\vec l,\alpha}(t)=-\sum_{\vec l',\alpha'}\vec\Phi_{\alpha,\alpha'}\left(\vec R_\vec l-\vec R_\vec l'\right)\cdot\vec u_{\vec n',\alpha'}(t)
\end{equation}
 Aquesta equació representa un sistema  d'oscil·ladors harmònics acoblats, on $\alpha$ i $\alpha'$ fan referència al àtoms de la base considerats, l'índex vectorial $\vec l$ (que sol aparèixer en la literatura com $n$) índica el nuc considerat i $\vec R_\vec l$ és el vector de translació cristal·lina.
 
\vdots 




\subsubsection{Matriu dinàmica}

La matriu dinàmica és la magnitud central de la dinàmica reticular: les freqüències dels fonons es calculen a partir dels valors propis de la matriu dinàmica:

\begin{equation}
\sum_{\alpha\prime}D_{\alpha\alpha\prime}(\vec q)\cdot\vec e_{\alpha\prime}(\vec q)=\omega^{2}\vec e_{\alpha}(\vec q)
\end{equation}   

Per tant, les freqüències $\omega$ com funció del vector d'ones $\vec q$ del fonó són solució de l'equació secular:

\begin{equation}
\det\left|\frac{1}{\sqrt{M_\alpha M_{\alpha\prime}}}D^{ij}_{\alpha\alpha\prime}\left(\vec q\right)-\omega^2\left(\vec q\right)\right|=0 
\end{equation}

on $M_{\alpha}$ es la massa de l'àtom $\alpha$ y la matriu dinàmica ve definida por:

\begin{equation}
D_{\alpha,\alpha\prime}^{i,j}=\frac{\partial^2 E}{\partial u_{\alpha}^{*i}(\vec q)\partial u_{\alpha\prime}^{j}(\vec q)}
\label{eq:Matriz_Dinámica}
\end{equation}

on $u_{\alpha}^{i}$ representa el despla\c{c}ament de l'àtom $\alpha$ en la direcció $i$.

La segona derivada de l'energia de l'equació \ref{eq:Matriz_Dinámica} correspon al canvi en la for\c{c}a que actua sobre l'àtom $\alpha\prime$ en la direcció $j$ quan es despla\c{c}a l'àtom $\alpha$ en la direcció $i$

\begin{equation}
D_{\alpha\alpha\prime}^{ij}(\vec q)=\frac{\partial}{\partial u^{*\alpha}_{i}}F^{j}_{\alpha'}(\vec q)
\end{equation}

\vdots

Passem ara a descriure el sistema que volem estudiar


\newpage

\section{Descripció del cristall de BN}

Donat que el càlcul dels modes de vibració comen\c{c}a per establir la geomeria del cristall en equilibri, comprobem amb les dades proporcionades que el $BN$ monocapa es tracta d'un cristall bidimensional hexagonal de base diatómica, la ce\l.la unitat del qual ve donada per (dades proporcionades):

\begin{sagesilent}
var('a', domain='positive')
a_1=a*vector([1,0])
a_2=a*vector([-1/2,sqrt(3)/2])
\end{sagesilent}

\begin{equation}
\vec a_1=\sage{a_1}\qquad\vec a_2=\sage{a_2} 
\end{equation}

Podem comprobar que efectivament es tracta de una ce\l.la hexagonal, ja que els seus vectors primitius formen un angle de $\sage{arccos(a_1*a_2/(norm(a_1)*norm(a_2)).simplify())}$ \SI{}{\radian}.

Les posicions atòmiques d'equilibri en la ce\l.la unitat són:

\begin{sagesilent}
r_B=1/3*a_1+2/3*a_2; r_N=2/3*a_1+1/3*a_2
cela=arrow((0,0),(a_1/a),color="blue")+\
      arrow((0,0),(a_2/a),color="red")+\
      line([(a_1/a),(a_1/a+a_2/a)],linestyle="--",color="red")+\
      line([(a_2/a),(a_2/a+a_1/a)],linestyle="--",color="red")+\
      point(r_B/a, size=120,color="blue")+\
      point(r_N/a, size=100,color="red")
\end{sagesilent}

\begin{equation}
\vec d_B=\frac{1}{3}\vec{a_1}+2\vec{a_2}\qquad
\vec d_N=\frac{2}{3}\vec{a_1}+\frac{1}{3}\vec{a_2} 
\end{equation} 



Tal com s'indica en la introducció teòrica, enumere les ce\l.les amb un índex vectorial $\vec l=\left(l_1, l_2\right)$, on $l_1$ i $l_2$ són dos nombres enters.

Les posicions del nucs de la xarxa són $\vec R_{\vec l}=l_1\vec{a}_1+l_2\vec{a}_2$.

\begin{sagesilent}
nucs=points([l_1*a_1/a+l_2*a_2/a for l_1 in range(-3, 4) for l_2 in range(-3,4)], size=40, color="blue", frame=False)

xarxa=nucs+\
line([(0,0),(a_1/a)],color="red")+\
line([(0,0),(a_2/a)],color="red")+\
line([(a_1/a),(a_1/a+a_2/a)],color="red")+\
line([(a_2/a),(a_2/a+a_1/a)],color="red")+\
line([(a_2/a),(a_1/a+2*a_2/a)],linestyle="--")+\
line([(a_1/a+2*a_2/a),(2*a_1/a+2*a_2/a)],linestyle="--")+\
line([(2*a_1/a+2*a_2/a),(2*a_1/a+a_2/a)],linestyle="--")+\
line([(2*a_1/a+a_2/a),(a_1/a)],linestyle="--")

#Posicions d'equilibri dels àtomos
def RB(l_1,l_2):
    return (l_1*a_1+l_2*a_2+r_B)

def RN(l_1,l_2):
    return (l_1*a_1+l_2*a_2+r_N)

AtomsB=points([RB(l_1,l_2)/a for l_1 in range(-3, 4) for l_2 in range(-3,4)],size=20,color='blue')

AtomsN=points([RN(l_1,l_2)/a for l_1 in range(-3, 4) for l_2 in range(-3,4)],size=20,color='red')

\end{sagesilent}

\begin{figure}[h]
\centering
\begin{subfigure}[b]{0.3\textwidth}
\centering
\sageplot[width=\textwidth]{xarxa, figsize=3}
\caption{Xarxa de nucs}
\end{subfigure}
\begin{subfigure}[b]{0.3\textwidth}
\centering
\sageplot[width=\textwidth]{cela, figsize=3}
\caption{Ce\l.la unitat}
\end{subfigure}
\begin{subfigure}[b]{0.3\textwidth}
\centering
\sageplot[width=\textwidth]{AtomsB+AtomsN, figsize=3}
\caption{Cristall}
\end{subfigure}
\end{figure}

Per construir la matriu dinàmica necessitem com a pas previ classificar el átoms del cristall segons la seua distància als àtoms de la ce\l.la unitat, ja que els clasificarem com primers, segons, tercers ... veïns segons aquesta distància i els asignarem un tensor de constants de forces que dependrá de a quina familia de veïns pertanyen. 

\begin{sagesilent}
#import pandas as pd
var('q_x, q_y'); assume(q_x, q_y, 'real'); q=vector([q_x,q_y])


## Parametros de la red, de la celdilla y del cristal
## Vector R_l (vector de traslación primitivo)
def R_l(l_1,l_2):
    return l_1*a_1+l_2*a_2

## Vector de posición de los átomos del cristal (en equilibrio)
def R_alpha_l(alpha,l_1,l_2):
    if alpha == 1:
        return l_1*a_1+l_2*a_2+r_B

    elif alpha == 2:
        return l_1*a_1+l_2*a_2+r_N

    else:
        print("Error, alpha solo puede ser 1 o 2 ")

## Vector unitario que une uno de los átomo alphaprima con el átomo considerado alpha, l_1,l_2
def R_hat(alphaprima,alpha,l_1,l_2):
    if (R_alpha_l(alpha,l_1,l_2)-R_alpha_l(alphaprima,0,0)).norm()>0:
        return (R_alpha_l(alpha,l_1,l_2)-R_alpha_l(alphaprima,0,0))/(R_alpha_l(alpha,l_1,l_2)\
                                                                 -R_alpha_l(alphaprima,0,0)).norm()
    else:
        return (R_alpha_l(alpha,l_1,l_2)-R_alpha_l(alphaprima,0,0))

# Distancia entre el átomo alphaprime y su vecino alpha situado en la celdilla l_1,l_2
def distancia(alphaprime,alpha,l_1,l_2):
    return (R_alpha_l(alpha,l_1,l_2)-R_alpha_l(alphaprime,0,0)).norm()/a    

def fase(l_1,l_2):
    return exp(I*q*R_l(l_1,l_2))

#Genero una lista con la distancia de cada átomo a los átomos de la celdilla unidad 
#def valores_atomos(l_1, l_2):
#    return [(k, m, i, j, R_hat(k, m, i, j), distancia(k,m,i,j)) for k in [1,2] for m in [1,2]  \
#            for i in range(-l_1,l_1+1) for j in range(-l_2,l_2+1)]

## Construyo un DataFrame de pandas con la información necesaria para identificar a los primeros, segundos, ... vecinos, según su distancia a cada uno de los átomos de la celdilla unidad
#columnas = [r"$\alpha\prime$",r"$\alpha$",r"$l_1$", r"$l_2$", r"$\hat R$",'Distancia']

#def lista_atomos(l_1, l_2):
#    return pd.DataFrame(valores_atomos(l_1,l_2),columns=columnas).sort_values(['Distancia',r"$\alpha\prime$"], ascending=[True,True])

## Mostramos el dataframe como una tabla
#table(lista_atomos(2,2).to_html(index=False))
\end{sagesilent}


\subsection{Primera zona de Brillouin}
Passe ara a mostrar la xarxa $2\pi$-recíproca associada a la xarxa de nucs del cristall de $BN$, la primera zona de Brillouin y calcular els punts de màxima simetria de la primera zona de Brillouin:

Els vectors primitivos de la xarxa $2\pi\text{-recíproca}$ són:

\begin{equation}
\label{eq:11}
\vec b_1=\frac{2\pi}{a}\left(\hat k_{x}+\frac{1}{\sqrt{3}}\hat k_{y}\right)\quad \vec b_2=\frac{2\pi}{a}\left(\frac{2}{\sqrt{3}}\hat k_y\right)
\end{equation} 

i la xarxa $2\pi$-recíproca és:

\begin{sagesilent}
import numpy as np
import matplotlib.pyplot as plt
import pybinding as pb

a_1_rec=np.array([1,1/sqrt(3)])
a_2_rec=np.array([0,2/sqrt(3)])
  
def R_l_rec(l_1,l_2):
    return 2*np.pi*l_1*a_1_rec+2*np.pi*l_2*a_2_rec 

reddenudos=np.array([R_l_rec(l_1,l_2) for l_1 in range(-5, 5) for l_2 in range(-3,4)])

  
x = reddenudos[:,0]
y = reddenudos[:,1]
plt.plot(x,y,"o")
plt.axis('scaled')
plt.savefig("../Grafiques/Xarxa2pireciproca.jpg")
plt.close()

#from math import sqrt, pi

pb.pltutils.use_style()
from pybinding.repository.graphene.constants import a_cc, t

a = 56 / 55 * a_cc  # ratio of lattice constants is 56/55, in [nm] unit cell length
#t in [eV] hopping energy

vn = -1.4  # [eV] nitrogen onsite potential
vb = 3.34  # [eV] boron onsite potential

# create a simple 2D lattice with vectors a1 and a2
lattice = pb.Lattice(a1=[a, 0], a2=[-a/2, np.sqrt(3)*a/2])
lattice.add_sublattices(
('B', [0, np.sqrt(3)*a/3], vb), # add an atom called 'A' at position [0, 0]
('N', [a/2,np.sqrt(3)*a/6] ,vn )
)

lattice.add_hoppings(
        ([ 0,  0], 'B', 'N', t),
        ([ -1, 0], 'B', 'N', t),
        ([0,  1], 'B', 'N', t)
    )


lattice.plot()
lattice.plot_brillouin_zone()
plt.savefig("../Grafiques/1aBZ.jpg")
plt.close()
reset('a')
var('a', domain='positive')
\end{sagesilent}

\begin{figure}[h]
\centering
\begin{subfigure}[b]{0.4\textwidth}
\includegraphics[width=8cm]{../Grafiques/Xarxa2pireciproca.jpg}
\subcaption{Xarxa en l'espai $2\pi$-recíproc}
\end{subfigure}
\begin{subfigure}[b]{0.4\textwidth}
\includegraphics[width=6cm]{../Grafiques/1aBZ.jpg}
\subcaption{Primera zona de Brillouin}
\end{subfigure}
\end{figure}





\subsection{Tensor de constants de forca i matriu dinàmica}

Una vegada descrit el nostre sistema físic anem a obtindre la matriu dinàmica d'aquest, ja que com hem vist els seus valors propis, $\omega^2$, ens donen la freqüència de propagació de cadascun dels modes.

Obtesses les posicions dels àtoms y classificats estos com primers, segons, etc. veïns, segons la distància al respectiu àtom de la ce\l.la $\vec 0$, procedim a calcular la contribució a la matriu dinàmica de cadascun dels àtoms, per la qual cosa necessitem conèixer el tensor de constants de for\c{c}a que correspon a la interacció de cada àtom amb el seu n-èssim veí.

La forma general del tensor de forces d'un $n$-èssim veí és de la forma \cite{wirtz04_phonon_disper_graph_revis}:


\begin{equation}
C_n=\begin{pmatrix}
\phi_n^{r}&\xi_n &0\\
-\xi_n & \phi_n^{ti} & 0 \\
0 & 0 & \phi_n^{to}
\end{pmatrix}
\label{eq:tensordeforces}
\end{equation}

on el sistema de coordenades se elegeix de manera que $x$ és la coordenada longitudinal (en la línia que conecta els dos àtomos), $r$; $y$ la coordenada transversal en el planol, $ti$, i $z$ la coordenada perperndicular al planol $to$. L'estructura diagonal a blocs de la matriu reflexa el fet que las vibracions interplanars y les de fora de ,$to$, (en la direcció $z$) estan completament desacoblades.



Anem a supossar (per simplificar) que un despla\c{c}ament longitudinal (radial, que estarà contés en el planol del cristall) o transversal (tangencial, siga en el planol o perpendicular al planol) sols genera una for\c{c}a radial o transversal, es a dir, $\xi_n=0$ tal y com es realitza \cite{Balkanski_2000}. Esta aproximació es coneix com la aproximación \textit{4NNFC},% i es necessita considerar fins els cuarts veïns per donar compte dels resultats experimentals.

%\missingfigure{Ací va imatge mostrant el cristall i les forces}


%\begin{figure}[h]
%\centering
%\includegraphics[width=40mm,height=40mm]{example-image-a}
%\caption{Caption}
%\end{figure}


%Una altra aproximació al problema trobada en la literatura sobre els fonons del grafé és la coneguda com el modelo \textit{VFF} (\textit{valence-force field}), la qual determina els paràmetros de la matriu en l'equació \ref{eq:tensordeforces} introduïnt \textit{constants de moll} que determinen el canvi en l'energia potencial segons diferents deformacions; una introducción a esta aproximació es troba en l'annexe de la referència (cite aizawa90 bond soften monol graph formed). Amb aquesta aproximació es necesiten menys paràmetres (\textit{constantes de fuerza}) per  obtindre una qualitat similar a la parametrizació \textit{4NNFC} (cite wirtz04 phonon disper graph revis).

Por tant, com primera aproximació anem a considerar que el tensor de constants de for\c{c}a  d'un àtom classificat com $n$-èssim veí, situat en la direcció $\hat x$ respecte del átom de la ce\l.la $\vec 0$, té la forma diagonal:

\begin{equation}
\label{eq:rotacio}
C_n(0)=\begin{pmatrix}
\phi_n^r& 0 &0\\
0 & \phi_n^{ti} & 0 \\
0 & 0 & \phi_n^{to}
\end{pmatrix}
\end{equation}

i per tant el tensor de constants de for\c{c}a de cadascun dels $n$-èssims veïns reals l'obtenim rotant la matriu de l'equació \ref{eq:rotacio}:

\begin{equation}
 C_n(\theta)=U^{-1}(\theta)C_n(0)U(\theta)
\end{equation}


on $\vec U(\theta)$ ve donada per:
\begin{equation}
\vec U(\theta)=
\begin{pmatrix}
\cos(\theta)  & \sin(\theta) & 0 \\
-\sin(\theta) & \cos(\theta) & 0  \\
0             & 0            & 1
\end{pmatrix}
\end{equation}

Un punt important és que quan considerem la interacció entre àtoms del mateix tipus (en el cas del BN és fácil discriminarlos, ja que són àtoms de distints elements els que conformen la base), tenim que considerar la contribució a la matriu dinámica de l'àtom que estamos considerant de la ce\.l.la $\vec 0$. Esta contribució podem derivar-la de la condició d'estabilitat, %(falkovsky08_symmet_const_phonon_disper_graph)
%(wirtz04_phonon_disper_graph_revis)
Si es desplaca el cristall com un tot no canvia l'energía potencial, cosa que implica s'ha de complir:


Podem tindre en compte altres simetries del cristall per determinar certes propietats del tensor de forces o de la seua transformada de Fourier, la matriu dinàmica (certes relacions entre les components ...) però per ara sols tindrem en compte que la matriu dinàmica es una matri hermítica, i per tant els seus valors propis, $\omega^2$ tenen que ser reals.

\vspace{5cm}

\begin{sagesilent}
#Angulo que forma el átomo considerado respecto al eje x

def angulo(alphaprima,alpha,l_1,l_2):
    if R_hat(alphaprima,alpha,l_1,l_2)[1] <0:
        return -acos(R_hat(alphaprima,alpha,l_1,l_2)*vector([1,0]))
    
    else:
        return acos(R_hat(alphaprima,alpha,l_1,l_2)*vector([1,0]))

#Matriz unitaria de rotación para cambio de ejes coordenados (entorno al eje z)

def U(theta):
    return matrix([[cos(theta),sin(theta),0], [-sin(theta), cos(theta),0],[0,0,1]])

#Matriz de fuerza para los primeros vecinos

var('M_B,M_N', domain='positive')
var('omega')

phi1rBN=var('phi1rBN',latex_name='\\phi_{1,r}^{BN}')
phi1tiBN=var('phi1tiBN',latex_name='\\phi_{1,ti}^{BN}')
phi1toBN=var('phi1toBN',latex_name='\\phi_{1,to}^{BN}',domain='real')

phi1rNB=phi1rBN; phi1tiNB=phi1tiBN; phi1toNB=phi1toBN


Phi_10__BN=1/sqrt(M_B*M_N)*Matrix([[phi1rBN,0,0],[0,phi1tiBN,0],[0,0,phi1toBN]])
Phi_10__NB=1/sqrt(M_N*M_B)*Matrix([[phi1rNB,0,0],[0,phi1tiNB,0],[0,0,phi1toNB]])


def Phi_1l__BN(alphaprima,alpha,l_1,l_2):
    return U(-angulo(alphaprima,alpha,l_1,l_2))*Phi_10__BN*\
           U(angulo(alphaprima,alpha,l_1,l_2))

def Phi_1l__NB(alphaprima,alpha,l_1,l_2):
    return U(-angulo(alphaprima,alpha,l_1,l_2))*Phi_10__NB*\
           U(angulo(alphaprima,alpha,l_1,l_2))

def D_1l_BN(alphaprima,alpha,l_1,l_2):
    return Phi_1l__BN(alphaprima,alpha,l_1,l_2)*fase(l_1,l_2)

def D_1l_NB(alphaprima,alpha,l_1,l_2):
    return Phi_1l__NB(alphaprima,alpha,l_1,l_2)*fase(l_1,l_2)

# Matriz de fuerza para los segundos vecinos

phi2rBB=var('phi2rBB',latex_name='\\phi_{2,r}^{BB}')
phi2tiBB=var('phi2tiBB',latex_name='\\phi_{2,ti}^{BB}')
phi2toBB=var('phi2toBB',latex_name='\\phi_{2,to}^{BB}',domain='real')

phi2rNN=var('phi2rNN',latex_name='\\phi_{2,r}^{NN}')
phi2tiNN=var('phi2tiNN',latex_name='\\phi_{2,ti}^{NN}')
phi2toNN=var('phi2toNN',latex_name='\\phi_{2,to}^{NN}',domain='real')

Phi_20__BB=1/M_B*Matrix([[phi2rBB,0,0],[0,phi2tiBB,0],[0,0,phi2toBB]])
Phi_20__NN=1/M_N*Matrix([[phi2rNN,0,0],[0,phi2tiNN,0],[0,0,phi2toNN]])

#A tener en cuenta: cuando consideramos el mismo tipo de átomos 
# (de la misma subred, no porque sean el mismmo elemento)

def Phi_2l__BB(alphaprima,alpha,l_1,l_2):
    return U(-angulo(alphaprima,alpha,l_1,l_2))*Phi_20__BB*\
           U(angulo(alphaprima,alpha,l_1,l_2))

def Phi_2l__NN(alphaprima,alpha,l_1,l_2):
    return U(-angulo(alphaprima,alpha,l_1,l_2))*Phi_20__NN*\
           U(angulo(alphaprima,alpha,l_1,l_2))

def D_2l_BB(alphaprima,alpha,l_1,l_2):
    return Phi_2l__BB(alphaprima,alpha,l_1,l_2)*fase(l_1,l_2)

def D_2l_NN(alphaprima,alpha,l_1,l_2):
    return Phi_2l__NN(alphaprima,alpha,l_1,l_2)*fase(l_1,l_2)


#Matriz de fuerza para los terceros vecinos

phi3rBN=var('phi3rBN',latex_name='\\phi_{3,r}^{BN}')
phi3tiBN=var('phi3tiBN',latex_name='\\phi_{3,ti}^{BN}')
phi3toBN=var('phi3toBN',latex_name='\\phi_{3,to}^{BN}',domain='real')
phi3rNB,phi3tiNB,phi3toNB=phi3rBN,phi3tiBN,phi3toBN
Phi_30__BN=1/sqrt(M_B*M_N)*Matrix([[phi3rBN,0,0],[0,phi3tiBN,0],[0,0,phi3toBN]])
Phi_30__NB=1/sqrt(M_N*M_B)*Matrix([[phi3rNB,0,0],[0,phi3tiNB,0],[0,0,phi3toNB]])
                   
def Phi_3l__BN(alphaprima,alpha,l_1,l_2):
    return U(-angulo(alphaprima,alpha,l_1,l_2))*Phi_30__BN*\
           U(angulo(alphaprima,alpha,l_1,l_2))

def Phi_3l__NB(alphaprima,alpha,l_1,l_2):
    return U(-angulo(alphaprima,alpha,l_1,l_2))*Phi_30__NB*\
           U(angulo(alphaprima,alpha,l_1,l_2))

def D_3l_BN(alphaprima,alpha,l_1,l_2):
    return Phi_3l__BN(alphaprima,alpha,l_1,l_2)*fase(l_1,l_2)

def D_3l_NB(alphaprima,alpha,l_1,l_2):
    return Phi_3l__NB(alphaprima,alpha,l_1,l_2)*fase(l_1,l_2)

#Matriz de fuerza para los cuartos vecinos
phi4rBN=var('phi4rBN',latex_name='\\phi_{4,r}^{BN}')
phi4tiBN=var('phi4tiBN',latex_name='\\phi_{4,ti}^{BN}')
phi4toBN=var('phi4toBN',latex_name='\\phi_{4,to}^{BN}',domain='real')
phi4rNB,phi4tiNB,phi4toNB=phi4rBN,phi4tiBN,phi4toBN

Phi_40__BN=1/sqrt(M_B*M_N)*Matrix([[phi4rBN,0,0],[0,phi4tiBN,0],[0,0,phi4toBN]])
Phi_40__NB=1/sqrt(M_N*M_B)*Matrix([[phi4rNB,0,0],[0,phi4tiNB,0],[0,0,phi4toNB]])

def Phi_4l__BN(alphaprima,alpha,l_1,l_2):
    return U(-angulo(alphaprima,alpha,l_1,l_2))*Phi_40__BN*\
           U(angulo(alphaprima,alpha,l_1,l_2))

def Phi_4l__NB(alphaprima,alpha,l_1,l_2):
    return U(-angulo(alphaprima,alpha,l_1,l_2))*Phi_40__NB*\
           U(angulo(alphaprima,alpha,l_1,l_2))

def D_4l_BN(alphaprima,alpha,l_1,l_2):
    return Phi_4l__BN(alphaprima,alpha,l_1,l_2)*fase(l_1,l_2)

def D_4l_NB(alphaprima,alpha,l_1,l_2):
    return Phi_4l__NB(alphaprima,alpha,l_1,l_2)*fase(l_1,l_2)

# Finalmente construimos la matriz dinámica "a capas"
# Con la tabla de la celda de codigo anterior comprobamos que para considerar 
# hasta los cuartos vecinos es suficiente con l_1,l_2=2
                   
D1BN, D1NB, D2BB, D2NN, D3BN, D3NB, D4BN, D4NB = (matrix(3) for i in range(8))
D01BN, D01NB, D02BB, D02NN, D03BN, D03NB, D04BN, D04NB= (matrix(3) for i in range(8))
for k in [1,2]:
    for m in [1,2]:
         for i in range(-2,3):
            for j in range(-2,4):
                if (k==1) & bool( distancia(k,m,i,j) == sqrt(3)/3 ):
                    D1BN += D_1l_BN(k,m,i,j)
                    D01BN += Phi_1l__BN(k,m,i,j)

                elif (k==2) & bool( distancia(k,m,i,j) == sqrt(3)/3 ):
                    D1NB += D_1l_NB(k,m,i,j)
                    D01NB += Phi_1l__NB(k,m,i,j)
                 
                elif (k==1) & bool( distancia(k,m,i,j) == 1):
                    D2BB += D_2l_BB(k,m,i,j)
                    D02BB += Phi_2l__BB(k,m,i,j)

                elif (k==2) & bool( distancia(k,m,i,j) == 1):
                    D2NN += D_2l_NN(k,m,i,j)
                    D02NN += Phi_2l__NN(k,m,i,j)

                elif (k==1) & bool( distancia(k,m,i,j) == 2*sqrt(3)/3 ):
                    D3BN += D_3l_BN(k,m,i,j)
                    D03BN += Phi_3l__BN(k,m,i,j)

                elif (k==2) & bool( distancia(k,m,i,j) == 2*sqrt(3)/3 ):
                    D3NB += D_3l_NB(k,m,i,j)
                    D03NB += Phi_3l__NB(k,m,i,j)
                    
                elif (k==1) & bool( distancia(k,m,i,j) == sqrt(7/3)):
                    D4BN += D_4l_BN(k,m,i,j)
                    D04BN += Phi_4l__NB(k,m,i,j)

                elif (k==2) & bool( distancia(k,m,i,j) == sqrt(7/3)):
                    D4NB += D_4l_NB(k,m,i,j)
                    D04NB += Phi_4l__NB(k,m,i,j)

# Tenemos en cuenta la contribución a la matriu dinámica de los átomos situados en 
# la celdilla 0
D2BB=D2BB-D01BN-D02BB-D03BN
D2NN=D2NN-D01NB-D02NN-D03NB                    
\end{sagesilent}

Tenim per tant que la matriu dinàmica serà una matriu $6x6$ hermítica. Però donat que les components en $z$ d'esta matriu es troben desacoblades podem tractar aquestes vibracions de manera independent.

\newpage

\subsection{Vibracions transversales fora de pla}
Donat que en el nostre model, per com hem construït la matriu dinàmica, les vibracions fora de pla són independents de les interplanars passem a estudiar-les.

\begin{sagesilent}
D1BN_zz=D1BN[2,2]
D1NB_zz=D1NB[2,2]

D2BB_zz=D2BB[2,2]
D2NN_zz=D2NN[2,2]

D3BN_zz=D3BN[2,2]
D3NB_zz=D3NB[2,2]
D4BN_zz=D4BN[2,2]
D4NB_zz=D4NB[2,2]

D_zz=Matrix([[D2BB_zz,D1BN_zz+D3BN_zz], [D1NB_zz+D3NB_zz,D2NN_zz]])
#valors_propis_D_zz=D_zz.eigenvalues()
\end{sagesilent}

\subsubsection{En $\Gamma (k_x=0, k_y=0)$}
Al punt $\Gamma$, ($q_x=0, q_y=0$), la matriu dinámica per a les vibracions fora de pla pren el valor:

\begin{sagesilent}
from periodictable import C, B, N, constants
u=constants.atomic_mass_constant*10**3 #para que este en CGS (y las const. de fuerza en dyn)

omega_Gamma_ZO=830 #cm-1
omega_Gamma_ZA=0

D_Gamma_zz=D_zz.subs(q_x=0,q_y=0) #,(M_B,B.mass*u),(M_N,N.mass*u)])
Eq1=(D_Gamma_zz.eigenvalues()[0]==omega**2).subs(omega=omega_Gamma_ZO)
\end{sagesilent}

\begin{equation}
\sage{D_Gamma_zz}
\end{equation}

i es seus valors propis son:

\begin{align*}
\omega_{ZO}^2(\Gamma)&=\sage{D_Gamma_zz.eigenvalues()[0]}\\
\omega_{ZA}^2(\Gamma)&=\sage{D_Gamma_zz.eigenvalues()[1]}
\end{align*}


\subsubsection{En $K$ ($q_x=4\pi/(3 a)$, $q_y=0$)}
Els valors propios de la matriu dinàmica en aques punt són:
\begin{sagesilent}
omega_K_ZO=605 #cm-1
omega_K_ZA=322
D_K_zz=D_zz.subs(q_x=4*pi/(3*a),q_y=0)
Eq1=(D_Gamma_zz.eigenvalues()[0]==omega**2).subs(omega=omega_Gamma_ZO)
solEq1=solve(Eq1, phi3toBN)
\end{sagesilent}

\begin{align*}
\omega_{ZO}^2(K)&=\sage{D_K_zz.eigenvalues()[0]}\\
\omega_{ZA}^2(K)&=\sage{D_K_zz.eigenvalues()[1]}
\end{align*}


Podem observar que en el cas del $BN$, a diferència del cas del grafé, obtenim 2 freqüències distintes al punt $K$ degut a que en la base tenim dos àtoms distints.

\begin{sagesilent}
sol=[]
Eq2=(D_K_zz.eigenvalues()[0]==omega**2).subs(omega=omega_K_ZO)
Eq3=(D_K_zz.eigenvalues()[1]==omega**2).subs(omega=omega_K_ZA)
sol.append(solve(Eq2.subs(solEq1),phi2toNN)[0])
sol.append(solve(Eq3.subs(solEq1),phi2toBB)[0])
\end{sagesilent}

\subsubsection{En el punt $M(q_x=\pi/a,q_y=\pi/(\sqrt 3 a)$}

\begin{sagesilent}
omega_M_ZO=635 #cm-1
omega_M_ZA=314

D_M_zz=D_zz.subs(q_x=pi/a,q_y=pi/(sqrt(3)*a))
#D_M_zz.eigenvalues()
#Podemos simplificar un poco la expresión obtenida para los valores propios en el punto  𝑀  (simplemente reescribiendo el argumento de la raiz cuadrada)

omegaM1cuadrado=-4*phi2toBB/M_B-4*phi2toNN/M_N-3/sqrt(M_B*M_N)*(phi1toBN+phi3toBN)-sqrt(M_B*M_N*(phi1toBN-3*phi3toBN)^2+(4*(M_N*phi2toBB-M_B*phi2toNN))^2)/(M_B*M_N)

#if bool(D_M_zz.eigenvalues()[0]==omegaM1cuadrado):
#    show(omegaM1cuadrado)

omegaM2cuadrado=-4*phi2toBB/M_B-4*phi2toNN/M_N-3/sqrt(M_B*M_N)*(phi1toBN+phi3toBN)+sqrt(M_B*M_N*(phi1toBN-3*phi3toBN)^2+(4*(M_N*phi2toBB-M_B*phi2toNN))^2)/(M_B*M_N)

#if bool(D_M_zz.eigenvalues()[1]==omegaM2cuadrado):
#    show(omegaM2cuadrado)
\end{sagesilent}

\begin{small}
\begin{align*}
\omega_{ZO}^2(M)&=\sage{omegaM1cuadrado}\\
\omega_{ZA}^2(M)&=\sage{omegaM2cuadrado}
\end{align*}
\end{small}

Comprobem que les expresions obtesses es redueixen a les que aparéixen en la ref: Falkowsky en el cas que es àtoms de la base foren iguals.

\begin{sagesilent}
Eq5=(omegaM1cuadrado==omega_M_ZO**2)
Eq6=(omegaM2cuadrado==omega_M_ZA**2)

sol1=(phi1toBN==n(solve(((Eq6-Eq5)**2).subs(solEq1), phi1toBN)[0].subs(sol[0], sol[1]).subs(M_B=B.mass, M_N=N.mass).rhs()))

sol2=sol[0].subs(M_N=N.mass)

sol3=sol[1].subs(M_B=B.mass)

sol4=solEq1[0].subs(M_B=B.mass, M_N=N.mass).subs(sol1)

\end{sagesilent}

De manera que los valores que obtenemos para las cuatro constantes de fuerza consideradas son:

\begin{align*}
&\sage{sol1}\qquad& \sage{sol2}\\
&\sage{sol3}\qquad& \sage{sol4}
\end{align*}

Y les relacions de dispersió per als modes $ZO$ i $ZA$ tenen la foma:

%\sagetexpause
\begin{sagesilent}
from pylab import loadtxt
#data = pd.read_csv('../Dades/freq.dat', header = None)
dades=loadtxt("../Dades/freq.dat")
ZAZO=\
list_plot(
    [real_part(n(sqrt(D_zz.subs(sol1, sol2, sol3, sol4, M_B=B.mass, \
        M_N=N.mass, a=1, q_x=n(x*pi), q_y=n(x*pi/sqrt(3))).simplify_full().eigenvalues()[1]))) \
        for x in srange(0,1,0.1)] +\
         [real_part(n(sqrt(D_zz.subs(sol1, sol2, sol3, sol4, M_B=B.mass, \
M_N=N.mass, a=1, q_x=n(pi*(1+x/3)), q_y=n(pi/sqrt(3)*(1-x))).simplify_full().eigenvalues()[1]))) \
        for x in srange(0,1,0.1)]+\
         [real_part(n(sqrt(D_zz.subs(sol1, sol2, sol3, sol4, M_B=B.mass, \
M_N=N.mass, a=1, q_x=n(4*pi/3*(1-x)), q_y=0).simplify_full().eigenvalues()[1]))) \
        for x in srange(0,1,0.1)]) + \
list_plot(
    [real_part(n(sqrt(D_zz.subs(sol1, sol2, sol3, sol4, M_B=B.mass, \
        M_N=N.mass, a=1, q_x=n(x*pi), q_y=n(x*pi/sqrt(3))).simplify_full().eigenvalues()[0]))) \
        for x in srange(0,1,0.1)]+
          [real_part(n(sqrt(D_zz.subs(sol1, sol2, sol3, sol4, M_B=B.mass, \
M_N=N.mass, a=1,q_x=n(pi*(1+x/3)), q_y=n(pi/sqrt(3)*(1-x))).simplify_full().eigenvalues()[0]))) \
        for x in srange(0,1,0.1)]+\
         [real_part(n(sqrt(D_zz.subs(sol1, sol2, sol3, sol4, M_B=B.mass, \
M_N=N.mass, a=1, q_x=n(4*pi/3*(1-x)), q_y=0).simplify_full().eigenvalues()[0]))) \
        for x in srange(0,1,0.1)]) \
     +line([(10,0),(10,1600)], color="black")+line([(20,0),(20,1600)], color="black")\
     +line([(30,0),(30,1600)], color="black", ticks=[[0.05,10,20,30], None], \
        tick_formatter = [[r'$\Gamma$', '$M$', '$K$', r'$\Gamma$'], None])+\
points(zip(dades[:524,0]/524*30, dades[:524,1]), color="red") +\
points(zip(dades[524:1048,0]/524*30, dades[524:1048,1]), color="brown") +\
points(zip(dades[1048:1572,0]/524*30, dades[1048:1572,1]), color="black") +\
points(zip(dades[1572:2096,0]/525*30, dades[1572:2096,1]), color="pink") +\
points(zip(dades[2096:2620,0]/525*30, dades[2096:2620,1]), color="steelblue") +\
points(zip(dades[2620:3144,0]/525*30, dades[2620:3144,1]), color="blue")

\end{sagesilent}

\begin{figure}[h]
\centering
 \sageplot{ZAZO,figsize=6}
\end{figure}

%\sagetexunpause


Passem ara a estudiar les vibracions dins del pla del cristall.

\newpage
\subsection{Vibracions dins del pla del cristall}
En este cas tractem amn una matriu $4\times 4$, i per tant tenim 4 valors propis ($\omega^2$)

\begin{sagesilent}
D1BN_xy=D1BN.matrix_from_rows_and_columns([0,1],[0,1])
D1NB_xy=D1NB.matrix_from_rows_and_columns([0,1],[0,1])

D2BB_xy=D2BB.matrix_from_rows_and_columns([0,1],[0,1])
D2NN_xy=D2NN.matrix_from_rows_and_columns([0,1],[0,1])
               
D3BN_xy=D3BN.matrix_from_rows_and_columns([0,1],[0,1])
D3NB_xy=D3NB.matrix_from_rows_and_columns([0,1],[0,1])
#D4BN_xy=D4BN.matrix_from_rows_and_columns([0,1],[0,1])
#D4NB_xy=D4NB.matrix_from_rows_and_columns([0,1],[0,1])

D_xy=block_matrix([[D2BB_xy, D1BN_xy+D3BN_xy],[D1NB_xy+D3NB_xy, D2NN_xy]])
\end{sagesilent}

\subsubsection{Al punt $\Gamma$}

Al punt $\Gamma$ obtinc 2 valors propis, de multiplicitat $2$ cadascun:
\begin{sagesilent}
D_Gamma_xy=D_xy.subs(q_x=0,q_y=0)
\end{sagesilent}

\begin{align*}
\omega_{1,2}^2(\Gamma)&=\sage{D_Gamma_xy.eigenvalues()[0]}\\
\omega_{3,4}^2(\Gamma)&=\sage{D_Gamma_xy.eigenvalues()[3]}
\end{align*}

El programa (escrit en python emprant \textcolor{blue}{sagemath}, que per a la diagonalització de matrius empra el programa \textcolor{red}{maxima}) no consigueix trobar un resultat analític \textit{sencill} per als valors propis de la matriu dinàmica als punts $M$ i $K$, de manera que anem a intentar obtindre algun resultat aproximat realitzant algunes simplificacions; en particular anem a considerar:

\begin{itemize}

\item Les masses dels àtoms són iguals $M_N=M_B$
\item Les constants de for\c{c}a entre àtoms del mateix tipus són també iguals: $\phi_{2,r}^{NN}=\phi_{2,r}^{BB}$, $\phi_{2,ti}^{NN}=\phi_{2,ti}^{BB}$ 
\end{itemize}

Emprant aquestes aproximacions (més endavant veurem que no fa falta fer tantes simplificacions, podem no igualar alguna de les dues constants anteriors), obtenim

\subsubsection{Al punt $M$}

Obtinc 4 valors propis distints, cosa que sembla raonable observant la gràfica de les dades proporcionades.
\begin{sagesilent}
D_M_xy=D_xy.subs(q_x=pi/a,q_y=pi/(sqrt(3)*a), phi2rNN=phi2rBB,phi2tiNN=phi2tiBB, M_N=M_B)
\end{sagesilent}

\begin{align*}
\omega_1^2(\Gamma)&=\sage{D_M_xy.eigenvalues()[0]}\\
\omega_2^2(\Gamma)&=\sage{D_M_xy.eigenvalues()[1]}\\
\omega_3^2(\Gamma)&=\sage{D_M_xy.eigenvalues()[2]}\\
\omega_4^2(\Gamma)&=\sage{D_M_xy.eigenvalues()[3]}
\end{align*}

\newpage

\subsubsection{Al punt $K$}

En este punt el programa en dona 3 valors propis, un d'ells amb multiplicitat $2$, i observant la gràfica de les dades proporcionades sembla també raonable que en primera aproximació un dels valors propis estiga degenerat (podem observar que tenim $2$ freqüències al punt $K$ molt pròxims entre ells)

\begin{sagesilent}
D_K_xy=D_xy.subs(q_x=4*pi/(3*a),q_y=0, phi2rNN=phi2rBB,phi2tiNN=phi2tiBB, M_N=M_B)\end{sagesilent}

\begin{align*}
\omega_1^2(\Gamma)&=\sage{D_K_xy.eigenvalues()[0]}\\
\omega_2^2(\Gamma)&=\sage{D_K_xy.eigenvalues()[1]}\\
\omega_{3,4}^2(\Gamma)&=\sage{D_K_xy.eigenvalues()[2]}
\end{align*}

Passem ara a intentar calcular els valors de les constants de for\c{c}a, emprant els valors de les freqüències dels modes de vibració en estes punts de la primera zona de Brillouin.
\newpage

\bibliography{TFG-Casimir}
\bibliographystyle{plain}

\end{document}
